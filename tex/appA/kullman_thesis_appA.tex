% ========== Appendix A
 
\chapter{Computing a Frontier's Hypervolume Indicator}
\label{chap:appAHypervolumeAlgo}
%N goes to M, O goes to mathcalM, P goes to Z, p goes to m
Given an efficient frontier $Z$ comprised of objective vectors $\mathbf{z} = [z_1,\ldots,z_M] \in Z$, this algorithm computes the volume $V$ of the objective space bounded by $Z$. This value is known as the hypervolume indicator. For more details on the hypervolume indicator, see \S \ref{chap:appCComparisonMetrics}.

The objectives are assumed to be normalized so that the objective space is the $M$-dimensional unit hypercube with the origin and the point $\vec{\mathbf{1}}$ defining the nadir objective vector and the ideal objective vector, respectively. That is, all objectives are assumed to be maximized and such that
$$ \forall i \in \{1,\ldots,M\} \quad z_i \in [0,1].$$

The algorithm projects the objective space into $M-1$ dimensions by eliminating the dimension associated with an objective $m \in \mathcal{M}$. We define the sub-dimensional objective set $\mathcal{L} = \mathcal{M} \backslash \{m\}$. It is assumed that $\mathbf{z} \in Z$ are sorted in descending order according to their $m$th component. The algorithm proceeds by sequentially adding solutions to the ($M-1$)-dimensional space, and calculating the contribution to the frontier volume $V$ as a product of the volume contribution in $M-1$ dimensions and $z_m$.

Let $\overbar{V_\mathbf{z}}$ be the ($M-1$)-dimensional volume contribution of solution $\mathbf{z}$. Further, let
$\mathbf{f} \in F$ be the non-dominated objective vectors in $M-1$ dimensions.
Compute the hypervolume $V$ as follows:

\begin{figure}[!ht]
\caption[Algorithm to compute the unary hypervolume indicator of a Pareto frontier]{Algorithm to compute the unary hypervolume indicator of a Pareto frontier.}
\begin{algorithmic}[1]

\State $V \gets 0$
\State $\overbar{V} \gets 0$
\State $F \gets \emptyset$

% Iterate over each solution
\ForAll{$\mathbf{z} \in Z$}

	\State $\overbar{V}_\mathbf{z} \gets \prod_{\ell \in \mathcal{L}} z_\ell - \overbar{V}$
		
	\ForAll{$\mathbf{f} \in F$}
		\If{$f_\ell < z_\ell \, \forall \ell \in \mathcal{L}$}
			\State $F \gets F \backslash \{\mathbf{f}\}$
		\EndIf
	\EndFor
	
	% iterate over subdimensions to "add back the sides"	
	\ForAll{$\ell \in \mathcal{L}$}
	
		\State $F_{\mathbf{z},\ell} := \set{\mathbf{f} \in F : f_\ell > z_\ell}$
		
		\State Sort $\mathbf{f} \in F_{\mathbf{z},\ell}$ in ascending order by $\ell$th component, $f_\ell$
		
		\State $v_i \gets z_\ell$
		\ForAll{$\mathbf{f} \in F_{\mathbf{z},\ell}$}
			\State $v_t \gets f_\ell$
			\State $\delta_\ell := v_t - v_i$
			\State $\overbar{V}_\mathbf{z} \gets \overbar{V}_\mathbf{z} + \delta_\ell \prod_{\lambda \in \mathcal{L} \backslash \{\ell\}} f_\lambda$
			\State $v_i \gets v_t$
		\EndFor
		
	\EndFor
	
	\State $F \gets F \cup \{\mathbf{z}\}$
	\State $\overbar{V} \gets \overbar{V} + \overbar{V}_\mathbf{z}$
	\State $V \gets V + z_m \overbar{V}_\mathbf{z}$
\EndFor


\end{algorithmic}
\end{figure}