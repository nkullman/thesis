\setcounter{page}{-1}
\abstract{%
DRAFT\\
Forests provide ecosystem services in concert with one another. Multi-objective optimization has been a successful approach used to determine forest management schemes that maximize the simultaneous provision of ecosystem services. Climate change is predicted to impact forests and their ability to provide ecosystem services; however, no studies have determined how the relationships between managed ecosystem services will change with climate. This study addresses that question using a scenario-based approach. I consider a study system in the Deschutes National Forest with competing objectives: providing habitat for the northern spotted owl, reducing fire hazard, and ensuring water quality of the watershed. I compare the tradeoffs among the objectives under three climate scenarios which vary in their intensity of assumed climate change.

I find that $\ldots$
}