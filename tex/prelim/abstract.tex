\setcounter{page}{-1}
\abstract{%
We present here a novel process to measure the conflict among objective functions within and across multi-objective systems. To do so, we introduce a new metric to quantify pairwise objective conflict. We also demonstrate new applications for two existing measures of conflict from evolutionary multi-objective optimization (EMO).

To demonstrate this quantification of conflict and the utility of our proposed pairwise objective conflict measure we cite two case studies. The first is a new case study on the the impact of climate change on the joint provision of forest ecosystem services in the Deschutes National Forest. For each of three climate scenarios, we quantify the total amount of conflict in the multi-objective system and we also compare the conflict across climate scenarios. We find that overall system conflict increases with increasing climate change severity and that climate change impacts both the individual and joint provision of ecosystem services. We also perform conflict quantification on an existing case study performed by T\'{o}th et al.\ (2013) in which joint provision of forest ecosystem services is considered under three different forest management sustainability certifications. We find differences in conflict for the various certification regimes yet similar pairwise objective conflict measures.

The results of our conflict quantification in these case studies shows that our proposed process and the new conflict metric are successful at quantifying and differentiating the amount of conflict within and across multi-objective systems and that they stand to serve as a useful tool for multi-objective decision making.