\setcounter{page}{-1}
\abstract{%
DRAFT\\
Forests provide a bounty to humans through ecosystem services such as wildlife habitat, recreation, and water and air purification. Forest managers seek to maximize the provision of ecosystem services and often do so for multiple ecosystem services simultaneously. While many studies predict that climate change will impact forests' ability to provide ecosystem services, no research has addressed the question of how climate change will impact the joint provision of ecosystem services. I address this question here in an attempt to better understand how the relationships between ecosystem services will change with climate. For example, how much additional fire hazard must be assumed in order to maintain an amount of habitat for a particular species. To study this question, I consider the growth of a forested area in the Deschutes National Forest under three climate scenarios of varying intensity. This area provides three competing ecosystem services whose joint provision is assessed under each of the climate scenarios: northern spotted owl habitat, water quality, and resistance to wildfire.
%
%will impact tradeoffs between ecosystem services. How will  This means that the future relatinoshiops between ecosystems serviecs are unknown. While current relationships may allow for the joint provide multiple ecosystem services simultaneously. Climate change is expected to alter forests' ability to provide ecosystem services, yet no study has addressed how this impacts the joint provision of ecosystem services. One such examOptimal ecosystem service provision is challenMany pairs of ecosystem services such as these are in conflict with one another - one ecosystem service must be sacrificed to achieve more of another.
%
%Climate change is predicted to to impact forests' ability to to provide ecosystem services, and will likely do so to varying extents and at varying speeds. Available habitat for some species may not change at all, while While habitat available for large mammals may improveThe joint provision of ecosystem , for example, wildlife habitat and Multi-objective optimization provides forest managers with a tool to  information about how the ecosystem services may be simultaneously achieved and how.  so it is necessary to understand the conflicts between these ecosystem services. For instance, what degree of fire risk must we endure in order to guarantee some amount of wildlife habitat? Climate change is predicted to impact forests and their ability to provide ecosystem services, and will likely do so heterogeneously. That is the speed, magnitude, and direction of change may differ between ecosystem services. While wildlife habitat for some species may  may , while some ecosystem services may degrade - such as a higher risk of fire - others may improve - such as carbon sequestered. it threatens to destabilize current relationships between ecosystem services - as  . Further, how does climate change impact this joint provision of ecosystem services and how do the conflicts between them change with climate? No study yet has addressed this question.
%
%I provide a solution here using a climate scenario-based approach. Using climate scenarios from the Fifth Assessment of the Intergovernmental Panel on Climate Change, I consider the evolution of a forested area that provides competing objectives and assess how climate impacts the joint provision of ecosystem services. The study area is in the Deschutes National Forest with competing objectives: providing habitat for the northern spotted owl, reducing fire hazard, and ensuring water quality of the watershed. I compare the tradeoffs among the objectives under three climate scenarios which vary in their intensity of assumed climate change.

I find that $\ldots$
}