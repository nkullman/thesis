% ========== Glossary

\chapter*{Glossary}      % starred form omits the `chapter x'
\addcontentsline{toc}{chapter}{Glossary}
\thispagestyle{plain}
%
\begin{glossary}

\item[climate projection] The IPCC defines a climate projection as a model-derived estimate of future climate. \textit{See} \textsc{climate scenario}\cite{ipcc2013Definition}.

\item[climate scenario] The IPCC defines a scenario as a coherent, internally consistent and plausible description of a possible future state of the world. Herein, I use this term synonymously with \textsc{climate projection}, since climate projections often underlie climate scenarios \cite{ipcc2013Definition}.

\item[cluster] Here, a set of contiguous forest stands whose combined area exceeds 200 ha

\item[ecosystem service] Benefits that people receive from ecosystems, divided into four categories: supporting, provisioning, regulating and cultural \cite{assessment2005ecosystems}. Examples include food, soil formation, water purification, carbon storage, recreation, and education.

\item[Pareto efficient] A solution to a multi-objective mathematical program is said to be Pareto efficient if no component of the solution can be improved without compromising at least one other component.

\item[Stand Density Index (SDI] Reineke's Stand Density Index is a measure of the stocking of a forest stand. See \cite{reineke1933perfecting}.
 
\end{glossary}