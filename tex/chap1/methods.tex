\section{Methods}
In response to stakeholder and public wants, forest planners often manage forests for multiple ecosystem services, such as wildlife habitat, recreation, goods production, aesthetics, and carbon sequestration. Such ecosystem services are commonly in conflict with one another, meaning that a forest planner cannot simultaneously maximize the provision of all ecosystem services. Instead, provision of some services must be sacrificed to enable achievement in others. Best compromises must be sought among bundled ecosystem services.

The Drink Planning Area in the Deschutes National Forest is one example of a region in which forest planners must consider the tradeoffs among ecosystem services when making management decisions.

\subsection{Study system}
\label{subsec:studyArea}
The study system for this analysis is the Drink Planning Area. The Drink is a 7056 ha area on the east slopes of the Cascade Mountain Range in the Deschutes National Forest (see Figure \ref{fig:drinkOverview}). Of the ecosystem services provided by the Drink, the US Forest Service has selected three for prioritization.

\begin{figure}[ht]
\centering
\includegraphics[width=.85\textwidth]{../images/DrinkMap_Overview}
\caption[Overview of the study system, the Drink Planning Area]{Overview of the study system, the Drink Planning Area (in purple), consisting of 7056 ha in the Deschutes National Forest.}
\label{fig:drinkOverview}
\end{figure}

The first is the protection of the northern spotted owl (NSO) (\textit{Strix occidentalis caurina}), a common, if controversial, indicator species in Pacific Northwest forests. Approximately 43\% of the Drink serves as habitat for the NSO (see Figure \ref{fig:drinkOwlAndWatershed}), and the USFS must protect this species since it is listed as threatened and therefore protected by the Endangered Species Act of 1973 \cite{congress1973endangered}.

The second objective is the reduction of the fire hazard rating across the Drink Area. Implementing silvicultural treatments to reduce the fire hazard of the Drink is critical because, in addition to providing NSO habitat, the municipal watershed for the cities of Bend, OR and Sisters, OR lies within the Drink (see Figure \ref{fig:drinkOwlAndWatershed}). These two cities have a combined population of approximately 90,000 people. Wildfires pose a threat to the their water supply, because wildfires cause soil water repellency, surface runoff, and debris torrents \cite{ice2004effects}.

\begin{figure}
\centering
\includegraphics[width=.5\textwidth]{../images/DrinkMap_NSOAndWatershed}
\caption[NSO Habitat and municipal watershed in the Drink Planning Area]{Location of the municipal watershed and the suitable NSO habitat in the Drink area at the beginning of the planning horizon (2015). Interior polygons are the 303 management units.}
\label{fig:drinkOwlAndWatershed}
\end{figure}

While the silvicultural treatments intend to provide long-term protection of the watershed, they also have the potential to introduce short-term increases in sediment delivery to the watershed \cite{o2005conceptual}. The minimization of these short-term peaks in sediment delivery is the the final ecosystem service prioritized by the USFS.

\subsection{Timeline and assessment of treatment efficacy}

Temporally, the case study in the Drink Planning Area consists of an 80 year simulation (2015 - 2095). All silvicultural treatments are performed in the first 40 years (2015 - 2055), divided into two 20-year planning periods. Spatially, the Drink is divided into 303 forest stands. Each stand may be treated in either period, neither, or both. Determining which treatment type to apply to a stand was done \textit{a priori} and is entirely dependent on silvicultural characteristics; the rules governing this assignment of treatment type can be found in the appendix, \S \ref{chap:appBTreatmentSpec}.

\begin{figure}
\centering
\includegraphics[width=.5\textwidth]{../images/DrinkMap_PAGs}
\caption[Plant association groups in the Drink Planning Area]{Plant association groups in the Drink Planning Area that were selected for potential treatments. Other plant association groups exist in the area but were not considered for treatment.}
\label{fig:drinkPAGs}
\end{figure}

To assess the treatments' long-term efficacy, the fire hazard rating of the Drink is measured at the end of the 80-year planning horizon in year 2095. The area of NSO habitat is assessed at the end of each planning period, years 2035 and 2055, to ensure that the application of treatments does not negatively impact the available habitat. Finally, at the time of treatment, the resulting short-term spikes in sediment delivery are measured. The time of treatment is assumed to be at the midpoint year in the planning period, years 2025 and 2045. A schematic of the planning horizon including the time of these events is shown in Figure \ref{fig:drinkPlanningHorizon}.

\begin{figure}
\centering
\includegraphics[width=.85\textwidth]{../images/Drink_PlanningHorizon_Sketch}
\caption[Planning horizon schematic]{The planning horizon used in the analysis spans the 80 year period from 2015 to 2095. Treatments may be performed in the first period, the second period, both, or neither. Treatments are assumed to be performed at the mid-point years of each period (black triangles). Sediment delivery is measured on treatment years. Stands' suitability for NSO habitat is measured at the end of the planning periods (gray triangles), and stands' fire hazard ratings are measured at the end of the planning horizon (white triangle).}
\label{fig:drinkPlanningHorizon}
\end{figure}

Competition exists between these bundled ecosystem services: fuel treatments in the watershed drive short-term peaks in sediment delivery and have the potential to reduce owl habitat; yet the prioritization of either NSO habitat or water quality alone entails fewer fuel treatments and increased fire hazard rating as a result. The forest planner, here the USFS, seeks a management plan that balances the provision of these bundled ecosystem services. The ability of the forest planer to decide on such a plan may be improved given a better understanding of the tradeoff relationships among the ecosystem services in the bundle. In this study, I seek to provide information on these tradeoff relationships and do so in a novel approach that also considers the impacts on climate change.

\subsection{Climate Scenarios Considered}
The IPCC uses a scenario-based approach to predicting climate change, presenting many models of future climates from research agencies around the world. There is no attempt to predict which future climate is most likely or quantify the probability of realization of any one scenario. I employ the same approach here in studying the potential impacts of climate change on tradeoff relationships among bundled ecosystem services. For a set of future climate scenarios, the joint provision of a bundle of ecosystem services is assessed.

The alternative future climates considered in this study are climate scenarios from the first working group (WG1) of the IPCC's Fifth Assessment (AR5) \cite{ipcc2013climate}. Given the large number of potential future climates considered by the IPCC (see the list of experiments considered in AR5 \cite{ipccListOfAR5Models}) combined with the computational complexity involved in the study of each one, I selected a small subset of three future climate scenarios for this analysis. Hereafter the scenarios are referred to as ``None'', ``Ensemble RCP 4.5'', and ``Ensemble RCP 8.5''.

The first scenario, ``None'', is the assumption of no climate change. While the number of studies incorporating climate change is increasing, this is still the assumption used for many modern studies such as Schroder (2013) \cite{schroder2016multi}, from which this study is derived. Because it has served as the basis for many studies and assumes a static environment resembling today's, the ``None'' climate scenario serves as a good control against which to compare the other two future climate scenarios.

As their names suggest, the second and third scenarios are ensembles. Each ensemble is an assembly of 17 global circulation models (GCMs) used in IPCC AR5. The selection of component GCMs in the ensembles was performed by the USFS's Climate-FVS \cite{dixon2002essential} team. The list of the 17 scenarios included in the ensemble can be found in Crookston (2016) \cite{ClimateModelsInFVSEnsemble}. Each component GCM has a corresponding climate surface which contains a vector of 35 climate parameters at over 11,000 global locations for three time periods. The climate surfaces for the ensembles were created by averaging the values of all component GCMs for each climate parameter and each time period for each location. The result is a climate surface that, while temporally sparse, is spatially robust. Such a configuration is suited for use in the Drink area given the area's variance in elevation and slow vegetation growth.

The two ensembles are comprised of the same 17 GCMs, but the assumed representative concentration pathways (RCP) in the component GCMs differ. The RCP indicates the additional radiative forcing in $W/m^2$ above pre-industrial levels, with higher values of forcing indicative of more severe climate change. The GCMs in the Ensemble RCP 4.5 scenario assume 4.5 $W/m^2$ of additional radiative forcing, and the GCMs in the Ensemble RCP 8.5 scenario assume 8.5 $W/m^2$ of additional radiative forcing.

These three scenarios were chosen for the analysis as they represent a range of predicted climate change severity, from a $0 \degree C$ warming by the year 2100 under the ``None'' scenario to a $2.6-4.8 \degree C$ warming under RCP 8.5 \cite{ipcc2013climate}.

\subsection{Determining tradeoff relationships between ecosystem services and climate scenarios}
\label{subsec:whyUsingMultiObjModel}
Each climate scenario is used to parameterize a multi-objective mathematical optimization model. These models determine the allocation of resources for optimal achievement of the objectives. Here, the resources are the application of fuel treatments and the objectives are the ecosystem services prioritized by the Forest Service: NSO habitat, fire hazard reduction, and short-term sediment delivery.

Solving each model generates a suite of management alternatives providing varying amounts in each ecosystem service. Comparing the ecosystem service achievements across the management alternatives reveals the tradeoff relationships among the ecosystem services. Since each model is parameterized according to a particular climate scenario, comparing the tradeoff relationships across models reveals how climate change impacts these tradeoffs among ecosystem services.

\subsection{The Multi-objective Optimization Model}
This section describes the multi-objective zero-one mathematical program to optimize the joint provision of ecosystem services in the Drink area. The model minimizes the fire hazard rating of the area, minimizes the peak sediment delivery occurring as a result of performing fuel treatments, and maximizes the minimum area of NSO habitat after treatment periods in the planning horizon.

\subsubsection{Notation}
The following notation is used in the model:
\paragraph{Parameters}
\begin{itemize}
\item \textbf{$i \in I$:} the set of all 303 stands in the Drink area
\item \textbf{$r \in R$:} the set of treatment schedule prescriptions:
	$$
	r =
	\begin{cases}
	1 &\text{ treatment applied in the first period (2015-2035)}\\
	2 &\text{ treatment applied in the second period (2035-2055)}\\
	3 &\text{ treatment applied in both periods}\\
	0 &\text{ no treatment applied in either period}
	\end{cases}
	$$
\item \textbf{$F_{i,r}$:} the area-weighted fire hazard rating of stand $i$ at the end of the planning horizon if prescribed to treatment schedule $r$
\item \textbf{$I_{\omega,t}$:} the set of stands that can qualify as NSO habitat at the end of planning period $t$
\item \textbf{$a_i$:} the area of stand $i$
\item \textbf{$e$:} the discount factor applied to NSO habitat that is less than 200 ha in size
\item \textbf{$j \in R_{i,t}$:} the set of treatment schedules such that stand $i$ qualifies as NSO habitat in planning period $t$
\item \textbf{$s_{i,t}$:} the contribution in tons of sediment delivered from performing fuel treatments on stand $i$ in planning period $t$
\item \textbf{$c \in C$:} the set of all clusters of stands whose combined area exceeds 200 hectares
\item \textbf{$i \in D_c$:} the set of all stands that comprise cluster $c$
\item \textbf{$c \in C_i$:} the set of all clusters that contain stand $i$
\item \textbf{$A$:} the maximum area in hectares that may be treated in either planning period
\item \textbf{$\ell$, $u$:} the upper and lower bounds, respectively, on the relative fluctuation in the area treated in periods 1 and 2
\end{itemize}

\paragraph{Decision Variables}
$$
x_{i,r} = \begin{cases}
1 &\text{ if stand $i$ is prescribed to treatment schedule $r$}\\
0 &\text{ otherwise}
\end{cases}
$$ 

\paragraph{Indicator Variables}
\begin{itemize}
\item \textbf{$q_{c,t} = 1$} if all stands in cluster $c$ qualify as NSO habitat in planning period $t$ and $q_{c,t} = 0$ otherwise
\item \textbf{$p_{i,t} = 1$} if in planning period $t$ stand $i$ is part of a cluster $c$ such that $q_{c,t} = 1$; $p_{i,t} = 0$ otherwise
\end{itemize}

\paragraph{Accounting Variables}
\begin{itemize}
\item \textbf{$S_t$:} the contribution in tons of sediment delivered from performing fuel treatments in planning period $t$
\item \textbf{$O_t$:} the amount of NSO habitat in hectares at the end of planning period $t$
\item \textbf{$H_t$:} the area in hectares treated in planning period $t$
\end{itemize}

\subsubsection{Parameterization}
The model was parameterized as follows:
\begin{itemize}
\item \textbf{$F_{i,r}$:} the metric for fire hazard rating used in this analysis originated in the work by Schroder \textit{et al.} \cite{schroder2016multi}. This metric was developed for the Drink area. It combines fire characteristics from Anderson's fuel models \cite{anderson1982aids} to assign a fire hazard rating. I expanded the rating system to include fuel models not present in Schroder \textit{et al.} See Table \ref{tab:firehazards}.

The stands' fuels and vegetation characteristics to determine the fire hazard rating were generated using the US Forest Service's Climate-Forest Vegetation Simulator (FVS). Input vegetation data to Climate-FVS came from the 2012 GNN structure map (\url{http://lemma.forestry.oregonstate.edu/data/structure-maps}) from Oregon State University's Landscape Ecology, Modeling, Mapping \& Analysis (LEMMA) group. Plots from the LEMMA database were mapped to the stands in the Drink area in order to produce tree and stand lists. These lists were used with Climate-FVS to simulate the stands' vegetation and fuels characteristics forward for the duration of the planning horizon under each climate scenario. Input climate data for Climate-FVS was obtained through the Climate-FVS climate data server \cite{climateFVSReadyData}.
\item \textbf{$I_{\omega,t}$:} the set of stands that qualify as NSO habitat at the end of a planning period $t$ are those that meet the following three criteria, as specified by the USFS:
	\begin{enumerate}
	\item elevation less than 1830 m
	\item the presence of trees with DBH no less than 76 cm
	\item canopy closure of at least 60\%
	\end{enumerate}
The elevation requirement was checked using a digital elevation model from the US Department of Agriculture's GeoSpatial Data Gateway; canopy closure and large tree requirements were determined using the simulated vegetation characteristics output from Climate-FVS.

To account for the NSO's large habitat requirements, stands must also be members of a cluster exceeding 200 ha in size, all of which meet the above three NSO habitat criteria. Stands not part of such a cluster have their contributions to owl habitat discounted by a factor of $e$.
\item \textbf{$e$:} the discount factor for sub-200 ha NSO habitat was set to $e = 0.5$ following the convention used in Schroder \textit{et al.} \cite{schroder2016multi}.
\item \textbf{$j \in R_{i,t}$:} each stand-treatment schedule combination is evaluated at the end of each planning period to determine its suitability as NSO habitat. Treatment schedules for which stand $i$ meets the criteria described above become members of the set $R_{i,t}$. 
\item \textbf{$s_{i,t}$:} the contributions of sediment delivery were determined using the Watershed Erosion Prediction Project (WEPP) online GIS tool \cite{frankenberger2011development}. This tool takes as input soil textures, treatment types, duration of simulation, and custom climate data. I obtained soil texture data for the Drink area from the USDA's Soil Survey Geographic (SSURGO) database. Treatment types are those specified in \S \ref{chap:appBTreatmentSpec}, and the years of simulation correspond to the treatment years in the model's planning horizon. The custom climate data are the same data described above for use with Climate-FVS, obtained through the Climate-FVS data server.
\item \textbf{$A$:} the maximum area that may be treated in either planning period was defined to be 6000 acres, or approximately 2428 ha
\item \textbf{$\ell$, $u$:} the relative fluctuation in the area treated in periods 1 and 2 was defined to be 20\%. That is, $\ell = 0.8$ and $u = 1.2$.
\end{itemize}

\begin{table}[!ht]
\centering
\resizebox{\textwidth}{!}{%
\begin{tabular}{l|c|crrr}
\multicolumn{1}{c|}{Fuel Model} & \multicolumn{1}{c|}{\textbf{Fire Hazard Rating}} & \multicolumn{1}{c}{Group} & \multicolumn{1}{c}{Flame length (m)} & \multicolumn{1}{c}{Rate of spread (m/hr)} & \multicolumn{1}{c}{Total fuel load (tons/ha)} \\ \hline
4*                              & \textbf{5}                                       & Shrub                     & 5.79                                 & 1508.76                                   & 32.12                                         \\
5                               & \textbf{4}                                       & Shrub                     & 1.22                                 & 362.10                                    & 8.65                                          \\
8                               & \textbf{1}                                       & Timber                    & 0.30                                 & 32.19                                     & 12.36                                         \\
9*                              & \textbf{2}                                       & Timber                    & 0.79                                 & 150.88                                    & 8.65                                          \\
10                              & \textbf{2}                                       & Timber                    & 1.46                                 & 158.92                                    & 29.65                                         \\
11*                             & \textbf{2}                                       & Logging Slash             & 1.07                                 & 120.7                                     & 28.42                                         \\
12                              & \textbf{4}                                       & Logging Slash             & 2.44                                 & 261.52                                    & 85.50                                         \\
13                              & \textbf{5}                                       & Logging Slash             & 3.20                                 & 271.58                                    & 143.57                                       
\end{tabular}%
}
\caption[Fire hazard ratings used in multi-objective model]{Fire hazard rating system used here, originally employed by Schroder \textit{et al.} \cite{schroder2016multi}.\\
* denotes fuel models not present in Schroder \textit{et al.}\\
The fuel model column refers to the Anderson fuel model ratings \cite{anderson1982aids}.}
\label{tab:firehazards}
\end{table}

\subsubsection{Formulation}
The formulation of the model is as follows:

\begin{align}
Minimize \quad & \notag\\
&\sum_{i\in I}\sum_{r\in R} F_{i,r} x_{i,r} \label{eqn:objFire} \\
&\max \{S_1,S_2\} \label{eqn:objSediment} \\
Maximize \quad & \notag\\
&\min \{O_1,O_2\} \label{eqn:objOwl}
\end{align}

Subject to:
\begin{align}
\sum_{i\in I_{\omega,t}} \left(a_i p_{i,t} + e a_i \left( \sum_{j \in R_{i,t}} x_{i,j}-p_{i,t} \right) \right) &= O_t \qquad \forall t \in \{1,2\} \label{eqn:constraintDefOwl}\\
\sum_{i\in I} \sum_{r\in 1,3} s_{i,1} x_{i,r} &= S_1 \label{eqn:constraintSediment1} \\
\sum_{i\in I} \sum_{r\in 2,3} s_{i,2} x_{i,r} &= S_2 \label{eqn:constraintSediment2} \\
\sum_{i \in D_c} \sum_{j \in R_{i,t}} x_{i,j} - |c| q_{c,t} &\ge 0 \qquad \forall t \in \{1,2\}, c \in C \label{eqn:constraintClusterTriggers} \\
\sum_{c \in C_i} q_{c,t} - p_{i,t} &\ge 0 \qquad \forall t \in \{1,2\}, i \in I_{\omega,t} \label{eqn:constraintPVarTriggers} \\
\sum_{r \in R} x_{i,r} &= 1  \qquad \forall i \in I \label{eqn:constraintOnePrescrip} \\
\sum_{i \in I} \sum_{r \in 1,3} a_i x_{i,r} &= H_1 \label{eqn:constraintAreaAcctg1} \\
\sum_{i \in I} \sum_{r \in 2,3} a_i x_{i,r} &= H_2 \label{eqn:constraintAreaAcctg2} \\
H_t &\le A \qquad \forall t \in \{1,2\} \label{eqn:constraintAreaRestr} \\
\ell H_1 - H_2 &\le 0 \label{eqn:constraintAreaFlucL} \\
-u H_1 + H_2 &\le 0 \label{eqn:constraintAreaFlucU} \\
x_{i,r}, p_i, q_c \in \{0,1\} \quad &\forall i \in I, r \in R, c \in C \label{eqn:constraintNonNeg}
\end{align}

Equations \eqref{eqn:objFire}-\eqref{eqn:objOwl} are the objective functions: equation \eqref{eqn:objFire} minimizes the cumulative fire hazard rating of the Drink area at the end of the 80-year planning horizon, equation \eqref{eqn:objSediment} minimizes the maximum peak in sediment delivery for the two planning periods, and equation \eqref{eqn:objOwl} maximizes the minimum NSO habitat available at the end of the planning periods. Equation set \eqref{eqn:constraintDefOwl} defines the amount of NSO habitat available at the end of the planning horizons. Note that if stand $i$ does not belong to a cluster of NSO habitat exceeding 200 hectares, then its area contribution to total NSO habitat is discounted by a factor of $e$. Equations \eqref{eqn:constraintSediment1} and \eqref{eqn:constraintSediment2} define the sediment delivered in planning periods one and two, respectively.

Inequality set \eqref{eqn:constraintClusterTriggers} controls the value of the cluster variables $q_{c,t}$ indicating clusters of suitable NSO habitat in each of the planning periods. Inequality set \eqref{eqn:constraintPVarTriggers} controls the value of the $p_{i,t}$ variables indicating stands' inclusion in NSO habitat clusters.

The set of equalities \eqref{eqn:constraintOnePrescrip} enforces the logical constraint that each stand must be prescribed to exactly one treatment schedule. Equations \eqref{eqn:constraintAreaAcctg1} and \eqref{eqn:constraintAreaAcctg2} are accounting constraints for the total area treated in each planning period, and inequalities \eqref{eqn:constraintAreaRestr} ensure that this area does not exceed the predefined per-period maximum. Inequalities \eqref{eqn:constraintAreaFlucL} and \eqref{eqn:constraintAreaFlucU} bound the fluctuation in treated area between the planning periods. Finally, constraint \eqref{eqn:constraintNonNeg} defines the decision and indicator variables as binary.

\subsection{Model solution}
In general, solving a bounded and non-degenerate multi-objective optimization problem with $N$ objectives produces a set of objective vectors (also called ``solutions'') $\mathbf{z} \in Z$ where $\mathbf{z}=\braket{z^1,\ldots,z^N}$. The set of solutions $Z$ is referred to as the Pareto-optimal frontier or efficient frontier or, simply, frontier. The solutions comprising an efficient frontier have the special relationship such that no component of a solution $\mathbf{z}^i$ can be improved upon without one of the other components $\mathbf{z}^j$ ($j \neq i$) degrading. This quality is known as Pareto efficiency. For example, this relationship in the current problem means that further reducing the value of fire hazard in a solution would result in either additional sediment deposits, a reduction of NSO habitat, or both.

Thus the efficient frontier provides information on the tradeoff structure that exists between ecosystem services. Parameterizing and solving the above model for each of the climate scenarios generates three frontiers: $Z_{\text{None}}$, $Z_{4.5}$, and $Z_{8.5}$ for the None, Ensemble RCP 4.5, and Ensemble RCP 8.5 scenarios, respectively. Since climate is the only thing that differs between the models and their resulting frontiers, comparing the frontiers provides insight into how climate impacts the tradeoff structures between the ecosystem services.

To solve the models, I wrote my own implementation of T\'{o}th's Alpha-Delta algorithm \cite{TothThesis} utilizing the IBM ILOG CPLEX optimization engine. The Alpha-Delta algorithm finds the optimal set $Z$ by iteratively slicing the $N$-dimensional objective space with a tilted $N-1$ dimensional plane. To derive the frontiers, I used an alpha parameter of $\alpha = .01$ and delta parameters of $\delta_{Hab} = 1$ ha and $\delta_{Sed} = 2$ tonnes for the NSO habitat and sediment delivery objectives, respectively.

\subsection{Comparing Tradeoffs under each Climate Change Scenario}
By parameterizing and solving the multi-objective model once for each climate scenario, I generated three efficient frontiers. To determine the impact of climate on tradeoffs in ecosystem services, I compared these frontiers and the level of conflict between the objectives within each of them. However, no standardized procedure exists for this task. In order to make comparisons at the frontier (climate scenario) level, I drew on methods used in the field of evolutionary multi-objective optimization (EMO). To address conflict between objectives within a frontier, I applied a method used for objective pruning in many-objective optimization and a variant of a method used in EMO.

\subsubsection{Comparing frontiers}
Researchers in the field of EMO develop algorithms to generate a set of non-dominated solutions that best represents the true Pareto-optimal frontier \cite{deb2001multi}. To test their algorithms, they solve a benchmark multi-objective optimization problem and compare their resulting frontiers to the known Pareto front for that problem \cite{knowles2002metrics}. There is no assurance of optimality of the solutions derived using these algorithms, so they require a means of comparing the resulting frontiers to determine if one algorithm produces a ``better'' non-dominated frontier than another. Zitzler et al. provide a review of comparison methods \cite{zitzler2003performance}. These methods aim to quantify certain traits about a frontier that can be used to measure their success in approximation of the true frontier.

My motivation in comparing frontiers is different from EMO in that, rather than comparing frontiers that result from solving identical models with varying methods, I compared frontiers that result from solving varying models (albeit with the same structure) with identical methods. The primary difference in the output of my approach compared to EMO is the assured optimality of the solutions in my frontiers. Because of this difference, not all comparison methods are applicable. For instance, the indicator for the number of Pareto points contained in the frontier does not make sense in my case, since all points on my frontiers are Pareto-optimal. Despite the difference, however, other comparison methods still have value. I chose a subset of these methods to compare my frontiers: the additive binary epsilon and binary hypervolume indicators, and the unary distance, additive unary epsilon, unary hypervolume, and unary spacing indicators.

Some methods for comparing frontiers require the normalization of the objective space. This is because the climate scenarios alter the bounds on the achievable values of the ecosystem services, resulting in frontiers whose objective spaces do not overlap.

In my analysis and in the definitions that follow, I chose the normalization such that all objectives are maximized, and each frontier is contained within the unit hypercube. That is, each objective is bounded between 0 and 1, yielding a frontier bounded by $[0,1]^N$. Defining the nadir solution $\mathbf{z}_{\text{nad}}$ of a frontier of points $z \in Z$ as the objective vector with components
\begin{align}
\mathbf{z}_{\text{nad}}^i = \inf_{z} \{ z^i \} \quad \forall 1 \le i \le N
\end{align}
and the ideal solution as the objective vector with components
\begin{align}
\mathbf{z}_{\text{ideal}}^i = \sup_{z} \{ z^i \} \quad \forall 1 \le i \le N
\end{align}
then under my normalization, the nadir solution is the origin and the ideal solution is the $N$-dimensional vector of ones $\mathbf{1}_N$.

The definitions of dominance terms used here are in Table \ref{tab:dominanceRelations}.

\begin{table}[ht]
\centering
\resizebox{\textwidth}{!}{%
\begin{tabular}{|p{.2\linewidth}|p{.1\linewidth}|p{.3\linewidth}|p{.1\linewidth}|p{.3\linewidth}|}
\hline
\textbf{Relation}           & \multicolumn{2}{c|}{\textbf{Solutions}}                                                                                                                                                                          & \multicolumn{2}{c|}{\textbf{Frontiers}}                                                                                                     \\ \hline
Strictly dominates & $\mathbf{z}_1 \succ \succ \mathbf{z}_2$ & $\mathbf{z}_1$ is better than $\mathbf{z}_2$ in all objectives                                                                                  & $Z_1 \succ \succ Z_2$ & Every solution in $Z_2$ is strictly dominated by at least one solution in $Z_1$                  \\ \hline
Dominates          & $\mathbf{z}_1 \succ \mathbf{z}_2$       & $\mathbf{z}_1$ is better than $\mathbf{z}_2$ in at least one objective and is not worse in any objective & $Z_1 \succ Z_2$       & Every solution in $Z_2$ is dominated by at least one solution in $Z_1$                           \\ \hline
Better             &                                         &                                                                                                                                                               & $Z_1 \rhd Z_2$        & Every solution in $Z_2$ is weakly dominated by at least one solution in $Z_1$ and $Z_1 \neq Z_2$ \\ \hline
Weakly dominates   & $\mathbf{z}_1 \succeq \mathbf{z}_2$     & $\mathbf{z}_1$ is at least as good as $\mathbf{z}_2$ in all objectives                                                                                        & $Z_1 \succeq Z_2$     & Every solution in $Z_2$ is weakly dominated by at least one solution in $Z_1$          \\ \hline
Incomparable       & $\mathbf{z}_1 || \mathbf{z}_2$          & Neither $\mathbf{z}_1$ nor $\mathbf{z}_2$ weakly dominates the other                                                                                          & $Z_1 || Z_2$          & Neither $Z_1$ nor $Z_2$ weakly dominates the other                                                         \\ \hline
\end{tabular}%
}
\caption[Dominance relationships for frontiers and solutions]{Definitions of dominance relationships between solutions and between frontiers, reproduced from Zitzler \textit{et al.} \cite{zitzler2003performance}.}
\label{tab:dominanceRelations}
\end{table}

\paragraph{Additive binary epsilon indicator $I_{\epsilon_+2}$} Given two frontiers, $Z_1$ and $Z_2$, the additive binary epsilon indicator is defined as \cite{zitzler2003performance}
\begin{align}
I_{\epsilon_+2} (Z_1,Z_2) = \inf_{\epsilon \in \mathbb{R}} \set{\forall \mathbf{z}_2 \in Z_2 \; \exists \mathbf{z}_1 \in Z_1 : \mathbf{z}_1 \succeq_{\epsilon_+} \mathbf{z}_2}
\end{align}
where $\succeq_{\epsilon_+}$ is the additive $\epsilon$-dominance relationship:
\begin{align}
\mathbf{z}_1 \succeq_{\epsilon_+} \mathbf{z}_2 \iff \epsilon + \mathbf{z}_1^i \ge \mathbf{z}_2^i \quad \forall 1 \le i \le N
\end{align}
Intuitively, $\epsilon$ is the minimum amount by which a frontier $Z_1$ must be translated such that every solution $\mathbf{z}_2 \in Z_2$ is ``covered''. See Figure \ref{fig:binaryEpsilon}. Positive values of $I_{\epsilon_+2} (Z_1,Z_2)$ indicate the presence of points $\mathbf{z}_2 \in Z_2$ that are not dominated by $Z_1$. Negative values of $I_{\epsilon_+2} (Z_1,Z_2)$ indicate that $Z_1$ strictly dominates $Z_2$ ($Z_1 \succ \succ Z_2$).

\begin{figure}[ht]
\centering
\includegraphics[width=.7\textwidth]{../images/BinaryEpsilonIndicator}
\caption[The additive binary epsilon indicator $I_{\epsilon_+2}$]{Depiction of the additive binary epsilon indicator $I_{\epsilon_+2}$ and the additive epsilon dominance relationship $\succeq_{\epsilon_+}$. In the figure,

\begin{minipage}{\linewidth}
  \begin{align*}
    I_{\epsilon_+2} (P,A) = -4 < 0 \qquad I_{\epsilon_+2} (P,B) = 0 \qquad I_{\epsilon_+2} (P,C) = 2 > 0
  \end{align*}
\end{minipage}

Region III is $\epsilon_+$-dominated for $\epsilon = 2$; region II is $\epsilon_+$-dominated for $\epsilon = 0$; region I is $\epsilon_+$-dominated for $\epsilon = -4$. Note that region II also encompasses region I, and region III encompasses region II.}
\label{fig:binaryEpsilon}
\end{figure}

\paragraph{Additive unary epsilon indicator $I_{\epsilon_+}$} I define the unary epsilon indicator as
\begin{align}
I_{\epsilon_+} (Z) = I_{\epsilon_+2} (Z,\mathbf{z}_{\text{ideal}})
\end{align}
That is, the additive unary epsilon indicator is identical to the additive binary epsilon indicator where the second frontier consists of a single point: the ideal solution for the first frontier.

This differs from the unary epsilon indicator traditionally used in EMO \cite{zitzler2003performance}. In EMO, the frontier is compared against a reference nondominated set. However, because my frontiers are optimal, there is no reference set against which to compare them.

\paragraph{Unary hypervolume indicator $I_{H1}$ and binary hypervolume indicator $I_{H2}$}
For every solution $\mathbf{z}_i$ in a frontier $Z$ define the hyperrectangle $r_i$ whose diagonal corners are the origin and the objective vector $\mathbf{z}_i = \braket{z^1,\ldots,z^N}$ (see Figure \ref{fig:frontierVolumes}). Then the unary hypervolume indicator of the frontier $Z$ is the $N$-dimensional volume of the union of all of the hyperrectangles corresponding to the solutions in $Z$:
\begin{align}
I_{H1} (Z) = \text{vol} \left( \bigcup_{i = 1}^{|Z|} r_i \right)
\end{align}
Then define the binary hypervolume indicator of two frontiers $Z_1$ and $Z_2$ as \cite{zitzler1999evolutionary}
\begin{align}
I_{H2} (Z_1,Z_2) = I_{H1} (Z_1 + Z_2) - I_{H1} (Z_2)
\end{align}
where $I_{H1} (Z_1 + Z_2)$ is the unary hypervolume indicator of the frontier consisting of the nondominated points in $Z = \{z \in Z_1 \cup Z_2\}$ . See Figure \ref{fig:binaryHypervolume}. The binary hypervolume indicator provides the volume of frontier $Z_1$ that is not contained within frontier $Z_2$. Larger values of $I_{H1}$ correspond to frontiers occupying larger amounts of the objective space. In a normalized objective space, $I_{H2}(Z_1, Z_2) > I_{H2}(Z_2, Z_1)$ indicates areas of less conflict between objectives in $Z_1$ than in $Z_2$.

\begin{figure}[ht]
\centering
\includegraphics[width=.85\textwidth]{../images/FrontierVolumesNo2DOutlines}
\caption[Hypervolume of Pareto frontiers]{Depiction of the hypervolumes of frontiers with two objectives (left) and three objectives (right).}
\label{fig:frontierVolumes}
\end{figure}

\begin{figure}[ht!]
\centering
\includegraphics[width=.7\textwidth]{../images/BinaryHypervolume}
\caption[Binary hypervolume indicator]{Depiction of the binary hypervolume indicator. The individual frontiers are shown in the top row: frontier $A$ (left) and frontier $B$ (right). The merged frontier $A+B$ is shown in bottom left - note the absence of points that were dominated when combined. Following the naming of regions as shown in the bottom right figure, the binary hypervolume indicator is equal to
\begin{minipage}{\linewidth}
  \begin{align*}
    I_{H2} (A,B) = \left(\text{area}_a + \text{area}_b + \text{area}_c \right) - \left( \text{area}_b + \text{area}_c \right) = \text{area}_a
  \end{align*}
\end{minipage}%
}
\label{fig:binaryHypervolume}
\end{figure}

I developed a custom algorithm to solve for the hypervolume idicators. The details of the algorithm may be found in \S \ref{chap:appAHypervolumeAlgo}.

\paragraph{Unary distance indicator $I_d$} The unary distance indicator measures the average distance from the frontier to the ideal solution:
\begin{align}
I_d = \frac{\sum_{\mathbf{z} \in Z} ||\mathbf{z}_{\text{ideal}} - \mathbf{z} ||}{N}
\end{align}
Smaller values of $I_d$ correspond to frontiers that are closer to the ideal solution, which may imply less conflict between objectives. This metric is analogous to the unary distance indicator more commonly used in EMO \cite{czyzzak1998pareto}. Where the metric used here measures the distance to the ideal solution, the traditional metric measures the distance to a reference Pareto frontier.

\paragraph{Unary Spacing Indicator $I_s$} The unary spacing indicator, or Schott's spacing metric \cite{schott1995fault}, computes the standard deviation of the distance between points in the frontier:
\begin{align}
I_s = \sqrt{\frac{1}{N-1} \sum_{\mathbf{z} \in Z} (d_z - \overbar{d})^2}
\end{align}
where
\begin{align}
d_z = \min_{\mathbf{y} \in Z, \mathbf{y} \neq \mathbf{z}} ||\mathbf{z} - \mathbf{y}||
\end{align}
and $\overbar{d}$ is the average over all $d_z$. In EMO, the spacing indicator provides a measure of an algorithm's ability to search the frontier space uniformly. Here, the spacing metric provides a measure of the flexibility afforded to the decision maker under each climate scenario, since smaller values of $I_s$ imply a higher density of solutions and greater flexibility.

\subsubsection{Quantifying conflict between objectives within a frontier}

The above methods provide frontier-level metrics of conflict and tradeoffs. To determine the degree of conflict between two objectives within a single frontier, we employ two techniques. The first is an approach used in many-objective optimization, and the second is a variant of the unary hypervolume indicator.

\paragraph{Pearson correlation coefficients} Given the increased difficulty in solving many-objective optimization problems \cite{khare2003performance}, researchers in this field seek to reduce the number of objectives considered in the model. To determine which objectives most strongly influence the shape of the frontier, they compute the correlation between each pair of objectives \cite{deb2005finding}. Objective pairs with strong negative correlation conflict with one another. To rank the relative conflict between objectives in each climate scenario, I compute their Pearson correlation coefficients:
\begin{align}
\rho_{X,Y} = \frac{\text{cov}(X,Y)}{\sigma(X)\sigma(Y)}
\end{align}
where, for objectives $x$ and $y$, $X$ and $Y$ are
\begin{align}
X = \{ \mathbf{z}^x_1, \mathbf{z}^x_2, \ldots, \mathbf{z}^x_{|Z|} \} \\
Y = \{ \mathbf{z}^y_1, \mathbf{z}^y_2, \ldots, \mathbf{z}^y_{|Z|} \}
\end{align}

\paragraph{Area of 2D frontier projection $A_{xy}$}
The second technique to measure the conflict between objectives within a frontier uses the unary hypervolume indicator. Given a frontier with objective vectors in $N$ dimensions, take two objectives $x$ and $y$, and project the $N$-dimensional frontier to the two-dimensional $xy$-plane. Remove solutions dominated in this projection, and compute the hypervolume indicator (which, in two-dimensions, is simply the area). See Figure \ref{fig:frontierCrossSection}. Larger values of $A_{xy}$ imply less conflict between objectives $x$ and $y$.

\begin{figure}[ht]
\centering
\includegraphics[width=\textwidth]{../images/CrossSection2D}
\caption[Area of 2D frontier projection]{Comparing conflict between objectives based on the area bounded by two-dimensional frontier projection. Left is the original frontier; middle shows the 2D projection of the frontier; right shows the projected frontier with all dominated solutions removed. Assuming both objectives are maximized, the larger the area bounded by the cross-sectional area, the less conflict between the objectives.}
\label{fig:frontierCrossSection}
\end{figure}