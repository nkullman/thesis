\section{Methods}
\subsection{Simultaneous Provision of Ecosystem Services}
\label{subsec:multiObjModel}
Aiming to determine how climate change may destabilize the relationships between managed ecosystem services, I followed the IPCC's approach of using scenario-based analyses. I selected three climate projections for consideration. The first scenario is the assumption of no climate change. This is the default assumption for many current studies such as !CITESVETLANASRESEARCH, from which this study is derived. I used this climate scenario as the control against which I  compared the ecosystem service tradeoffs observed in the other scenarios. The second and third scenarios are ensembles of climate models produced by research agencies recognized by the IPCC and assembled by the USFS !CITECLIMATEFVS. Details about the global circulation models (GCMs) included in the ensembles can be found in !CITETHECLIMATEMODELSDETAILSPAGE. The second scenario is an ensemble of models for Representative Concentration Pathway (RCP) 4.5 $W/m^2$, and the third scenario is the same ensemble of models for RCP 8.5 $W/m^2$. The RCPs indicate the additional radiative forcing (in $W/m^2$) above pre-industrial levels, with higher values of forcing indicative of more severe climate change. I chose these three scenarios because they represent a range of severity, from a $0 \degree C$ warming by the year 2100 under the control to a $2.6-4.8 \degree C$ warming under the third !CITEAR5SPM.

For each climate scenario I parameterized and solved a multi-objective mixed integer-linear mathematical program (MIP). The model is as follows:

%INSERT THE MODEL HERE
%INSERT THE MODEL DESCRIPTION JUST AFTER IT
\begin{align*}
Minimize \quad &\sum_{i\in I,r\in R} F_{i,r} x_{i,r} \\
Maximize \quad &\sum_{i\in I_\omega} \left(a_i p_i + e a_i \left( \sum_{j \in R_i} x_{i,j}-p_i \right) \right) \\
Minimize \quad &\max_t \{S_t\}
\end{align*}

Subject to the following constraints:
\begin{align}
\sum_{i\in I,r\in R} s_{i,r,t} x_{i,r} &= S_t \qquad \forall t \in T \\
\sum_{i \in D_c, j \in R_i} x_{i,j} - |c| q_c &\ge 0 \qquad \forall c \in C \\
\sum_{c \in C_i} q_c - p_i &\ge 0 \qquad \forall i \in I_\omega \\
\sum_{r \in R} x_{i,r} &= 1  \qquad \forall i \in I \\
\sum_{i \in I,r \in 1,3} a_i x_{i,r} &= H_1 \\
\sum_{i \in I,r \in 2,3} a_i x_{i,r} &= H_2 \\
H_1 &\le A \\
H_2 &\le A \\
\ell H_1 - H_2 &\le 0 \\
-u H_1 + H_2 &\le 0 \\
x_{i,r}, p_i, q_c \in \{0,1\} \quad &\forall i \in I, r \in R, c \in C
\end{align}													
Equation 1 defines the amount of sediment delivery in each time period. Equations 2 and 3 define account for the total area of NSO habitat, giving additional weight to habitat clusters. Equation 4 specifies that each subdivision of the study area must be assigned to a particular management prescription (even if the prescription is 'do nothing'). Equations 5 through 8 limit the total area of land that can be managed in any given year, and equations 9 and 10 limit the fluctuation in management in between years. Finally, equation 11 simply states that all variables are binary.

Using Climate FVS, I projected vegetation growth through the end of the 80 year planning horizon, which was used to parameterize the models' fuels and NSO coefficients. I received my initial vegetation data from Oregon State University's Landscape Ecology, Modeling, Mapping \& Analysis (LEMMA) group !CITELEMMA.

I used FVS's Fire and Fuels Extension (FFE) to obtain the average fuel model for each stand, which is used as a proxy for fire hazard. This proxy was chosen, because the higher the fuel model, generally, the larger the fuel loading. 

A forest stand's candidacy for NSO habitat is dependent on THREE CONDITIONS: the presence of at least one tree with DBH no less than SOMENUM centimeters, average stand elevation greater than SOMENUM meters, and contiguity with other such stands that, combined, they form a cluster at least 500 ha in size.


\subsection{Evaluating Climate Change Scenarios}
Solving each multi-objective model described in \S \ref{subsec:multiObjModel} produces a set of Pareto efficient points. Each point describes a prescription of management actions that, if followed, will produce an amount of NSO habitat, fire hazard, and sediment delivery as specified by the solution. While the magnitude of ecosystem service 

We drew on techniques used to compare sets of solutions in evolutionary algorithms.

\subsubsection{Computing a Frontier's Solution Spacing}
The spacing of solutions along the frontier provide a measure of flexibility for the decision maker. The more solutions 

\subsubsection{Computing a Frontier's Hypervolume Indicator}
To compare the tradeoff structure of each climate change scenario's corresponding Pareto frontier, I calculated the relative volume of the objective space bound by the frontier.  Computing such a volume for a two-dimensional frontier is trivial. Consider figure \ref{fig:2DFrontierVol}.
\begin{figure}[h]
  \centering
    \includegraphics[width=0.5\textwidth]{"../images/2DFrontierVolumeExample"}
  \caption{A two-dimensional frontier. The volume of this frontier may be computed by summing the areas of the rectangles shown.}
  \label{fig:2DFrontierVol}
\end{figure}
The reader can imagine a process to compute the volume whereby the frontier is divided into rectangles, as shown, and then summing the areas of these rectangles to get the total frontier volume.

Performing a similar computation in three and higher dimensions is less trivial and is an area of active research !CITESOMEONE. The higher-order volume computation is also often accomplished using Monte Carlo simulation !CITE SOMEONE.

I developed the following recursive algorithm to exactly compute the volume of an $n$-dimensional frontier for $n>2$.

Given a set of Pareto optimal solutions $\mathcal{P}$ to a multi-objective mathematical programming model with a set of objectives $O$ of cardinality $N := |O|$, this algorithm computes the volume $V$ of the objective space bounded by the Pareto frontier defined by the solutions $x \in \mathcal{P}$. The objectives are assumed to be normalized so that the objective space is the $N$-dimensional unit hypercube with the origin and the point $\vec{\mathbf{1}}$ defining the nadir objective vector and the ideal objective vector, respectively. That is, all objectives are assumed to be maximized.

We project the objective space into $N-1$ dimensions by eliminating the dimension associated with an (arbitrarily-chosen) objective $p \in O$. We define the set of objectives $\overbar{O} := O \backslash \{p\}$. It is assumed that $x \in \mathcal{P}$ are sorted in descending order according to $p$. The algorithm proceeds by sequentially adding solutions to the ($N-1$)-dimensional space, and calculating the contribution to the frontier volume as a product of the volume contribution in $N-1$ dimensions and its achievement in objective $p$.

Let
$\overbar{V_x}$ be the ($N-1$)-dimensional volume contribution of solution $x$ and
$x_p$ be the achievement of solution $x$ in objective $p$. Further, let
$F$ be the set of non-dominated solutions in $N-1$ dimensions.
We proceed to compute the $N$-dimensional volume of the frontier $V$ as follows.

\begin{figure}[!ht]
\caption{Algorithm to compute the volume of a Pareto frontier}
\begin{algorithmic}[1]

\State $V \gets 0$
\State $\overbar{V} \gets 0$
\State $F \gets \emptyset$

% Iterate over each solution
\ForAll{$x \in \mathcal{P}$}

	\State $\overbar{V}_x \gets \prod_{o \in \overbar{O}} x_{o} - \overbar{V}$
		
	\ForAll{$f \in F$}
		\If{$f_o < x_o \forall o \in \overbar{O}$}
			\State $F \gets F \backslash \{f\}$
		\EndIf
	\EndFor
	
	% iterate over subdimensions to "add back the sides"	
	\ForAll{$o \in \overbar{O}$}
	
		\State $F_{x,o} := \set{f \in F : f_o > x_o}$
		
		\State Sort $f \in F_{x,o}$ in ascending order by their $o$th component, $f_o$
		
		\State $v_i \gets x_o$
		\ForAll{$f \in F_{x,o}$}
			\State $v_t \gets f_o$
			\State $\delta_o :	= v_t - v_i$
			\State $\overbar{V}_x \gets \overbar{V}_x + \delta_o \prod_{\sigma \in \overbar{O} \backslash \{o\}} f_\sigma$
			\State $v_i \gets v_t$
		\EndFor
		
	\EndFor
	
	\State $F \gets F \cup \{x\}$
	\State $\overbar{V} \gets \overbar{V} + \overbar{V}_x$
	\State $V \gets V + x_p \overbar{V}_x$
\EndFor


\end{algorithmic}
\end{figure}

The result of the algorithm is a single metric for each frontier, known as the hypervolume indicator !CITESOMEONE. This metric is used in the field of Evolutionary Algorithms for MultiObjOpt. 