% ======== Introduction to Deschutes National Forest case study

\section{Introduction}
 
Forests play an important role in global ecological, social, and economic processes. They provide ecosystem services such as carbon storage, purification of water and air, wildlife habitat, recreation opportunities, and generate raw materials for goods such as food and lumber \cite{daily1997ecosystem}. In managed forests, the extent to which forests provide these services depends in part on management practices. Optimal forest management seeks to ensure the sustained provision of these ecosystem services \cite{cfrForestMgmt}.

Like other ecosystems, forests will undergo changes as a result of the changing climate. Researchers anticipate new spatial distributions of tree species \cite{iverson1998predicting}, increased sediment delivery to streams \cite{Goode20121}, and increasing disturbance regimes such as wildfires, drought, and insect infestation \cite{vose2012effects}. As this transformation occurs, forests' ability to provide ecosystem services will change. New growing conditions will alter timber production \cite{forschungsanstalten2009adaptation}. Novel temperatures and species compositions will alter forests' habitat suitability for inhabitants \cite{harding1997ecosystem}. Increased frequency of disturbance regimes will impact forests' ability to store carbon \cite{bonan2008forests} and provide wildlife habitat \cite{mckenzie2004climatic}. Water supplies that rely on forests' filtration capabilities may be impacted by the rising sediment levels predicted by \cite{Goode20121}.

Optimal forest management must consider the effects of the changing climate, because the time scale of forest development is of the same order as that on which climate change is predicted to operate \cite{ipcc2013climate}. Under alternative future climates, optimal forest management will likely differ \cite{linder2000developing}. Decisions that would once have resulted in optimal achievement of ecosystem services, now under different climatic conditions, may no longer do so. Without consideration of climate change, forest management plans may restrict forests' potential to provide ecosystem services most effectively.

-What's more, the relationships between managed ecosystem services - that is, the situations in which one ecosystem service may be sacrificed in order to achieve more of another - are unknown under new climates. Under non-optimal conditions, this has the potential consequence of sacrificing some amount of an ecosystem service with no benefit in return. Under optimal conditions, sacrifices in one ecosystem service can be used to achieve more of another. In the case of being unaware of these novel tradeoff relationships, this is no longer possible.
In addition to the magnitude of change in any one ecosystem service, it is also unknown how climate change will impact the relationships between ecosystem services. That is, the landscape-level simultaneous achievement of multiple ecosystem services.
For instance, consider an unfortunate hypothetical situation of a forest, which up until the year 2010 was prime habitat for bird species A. Climate change is quickly reducing the area of the forest that is suitable habitat for bird species A. The forest manager has always also managed the forest in order to generate some revenue from timber sales. 10 years ago, before considering climate cvhange, he ran a multiobj model that suggested he harvest from a small area near the northern perimeter of the forest. However, climate change is now reducing the area of the forest that bird species A can live in. The only part left now that it can comfortably inhabit are those areas on the northern perimeter. The forest manager is now unknowingly sacrificing a large percentage of the potential habitat for bird species A to maintain his harvest activities. Little does he know (bc he has not the advantage of using our research and knowing the tradeoff structure) that these two objectives are not really in such strong competition. If he would shift his harvest to the eastern stands, he would only sacrifice some small amount in harvest revenues (say a percent) and would achieve the maximum possible habitat for bird species A, enabling his rangeshift to his new northern areas.

 to as these are the areasin which climate change is rendering it incapable of providing habitat for long-time resident bird species A. This forest is also being managed to provide the ecosystem service of  currently provides habitat for bird species Abalancingthclimate It is currently unknown how climate will alter the 
in favor of another. However, that other ecosystem service maypoIn forests managed for multiple ecosystem services, decisions must be made to determine the allowable levels of 
Consider, for example that the decision is made within a stand at a certain point in the future to forgo timber revenue in favor of habitat for species A - however, it may be that the stand under a different climate does not meet habitat requirements for species A, and so an unnecessary sacrifice has been made in forgone revenue as a result. any longer for the for a certain Forest managers today have the . Even in the best case scenario - if they do still happen to be optimal - they may be missing out on beneficial tradeoff relationships

To determine which management practices will be optimal in the future, we must first understand how climate change will impact forests' ability to provide ecosystem services. For example, how will climate change alter the ? How many acres of forest will still qualify as suitable habitat for a particular species? Many studies have considered these questions !CITE SOME PEOPLE.

However, previous studies have addressed the impact of climate change on forest ecosystem services in isolation. Because forests provide these ecosystem services in concert with one another (see, for example, \cite{toth2009finding}), we must also understand how climate impacts the tradeoffs that exist among them. Consider the simultaneous provision of wildlife habitat, carbon storage, and resistance to wildfire. How does an increase in any one service alter our ability to acquire an amount of another? Relationships such as a marginal sacrifice in one service for substantial improvement in another may no longer exist under a new climate. To better ensure the sustained provision of ecosystem services, we must understand how these tradeoffs evolve as a function of climate.

Here, I use a watershed in the Deschutes National Forest as a case study to determine how climate change impacts optimal forest management and the changes in tradeoffs among ecosystem services.

TO TEST ALL THIS STUFF, I AM USING A STUDY AREA IN THE DESCHUTES NATIONAL FOREST, KNOWN AS THE DRINK AREA. IT IS THIS BIG AND IS DIVIDED INTO 303 FOREST STANDS. THE AREA CONTAINS THE WATERSHED FOR THE CITIES OF BEND AND SISTERS OREGON. IT IS COMPRISED OF OLD GROWTH AND NEW GROWTH AND SOME OTHER STUFF. IT IS FLAMMABLE. WE WANT TO REDUCE THE RISK OF LONGTERM, SEVERE DEGRADATION OF THE WATER SUPPLIES TO THESE CITIES THAT WOULD RESULT FROM A HIGH SEVERITY WILDFIRE. THIS IS OUR FIRST OBJECTIVE. WE WILL DO THIS BY PERFORMING FUEL TREATMENTS. BUT THESE FUEL TREATMENTS LEAD TO SHORTERM SPIKES IN SEDIMENT CONTENT IN THE WATERSUPPLY, WHICH WE AIM TO MINIMIZE. MINIMIZING THE SEDIMENT DELIVERY TO THE WATERSHED AS A RESULT OF THE TREATMENTS IS OUR SECOND OBJECTIVE. FINALLY, THE AREA IS HOME TO THE FEDERALLY PROTECTED NORTHERN SPOTTED OWL. OUR THIRD OBJECTIVE IS ENSURING MAXIMAL HABITAT FOR THE NSO. WE WANT TO TEST OUR ABILITY TO SIMULTANEOUSLY PROCURE THESE THREE ECOSYSTEM SERVICES IN THE LONGTERM. BY LONGTERM, I MEAN I WILL STUDY IT OVER AN 80 YEAR HORIZON FROM 2015-2095. ALL MANAGEMENT ACTIVITY WILL OCCUR DURING THE INITIAL 40 YEARS. BC THE AREA GROWS SLOWLY, WE MODEL THESE 40 YEARS IN TWO 20-YEAR PLANNING HORIZONS. THE MANAGEMENT ACTIONS THAT MAY BE PRESCRIBED ARE THINNING TREATMENTS (SEE APPENDIX \_ FOR TREATMENT PRESCRIPS) AND ARE DETERMINED APRIORI FOR EACH STAND AND TIME PERIOD COMBINATION. WE MEASURE THE SPIKE IN SEDIMENT DELIVERY AT THE TIME OF TREATMENT (YEARS 2025 AND 2045). WE MEASURE THE ACHIEVEMENT IN NSO HABITAT AND FIRE HAZARD AT THE END OF THE 80 YEAR PLANNING HORIZON. WE WILL DO THIS FOR EACH OF THREE DIFFERENT CLIMATE CHANGE SCENARIOS.

THE RESULTS WILL ENABLE US TO STUDY THE TRADEOFFS AMONG THESE THREE ECOSYSTEM SERVICES AND SEE HOW THEY VARY WITH CLIMATE CHANGE.