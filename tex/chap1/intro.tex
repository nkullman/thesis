% ======== Introduction to Deschutes National Forest case study

\section{Introduction}
 
Forests play an important role in global ecological, social, and economic processes. They provide ecosystem services such as carbon storage, purification of water and air, wildlife habitat, recreation opportunities, and generate raw materials for goods such as food and lumber \cite{daily1997ecosystem}. In managed forests, the extent to which forests provide these services depends on management practices. Optimal forest management seeks to ensure the sustained provision of these ecosystem services (!CITE bibtex'ed CFR source).

Similar to other ecosystems, forests will undergo changes as a result of the changing climate. We anticipate new spatial distributions of tree species \cite{iverson1998predicting}, increased sediment delivery to streams \cite{Goode20121}, and increasing disturbance regimes such as wildfires, drought, and insect infestation \cite{vose2012effects}. As this transformation occurs, forests’ ability to provide ecosystem services will change. NEW GROWING CONDITIONS MAY LEAD TO INC/DEC TIMBER PRODUCTION. TEMPERATURES MAY POSITION FORESTS AS HABITAT FOR MORE/FEWER SPECIES. Increased frequency of wildfires will impact forests’ ability to store carbon \cite{bonan2008forests} and provide habitat for wildlife \cite{mckenzie2004climatic}. Water supplies that rely on forests’ filtration capabilities may be impacted by the rising sediment levels predicted by \cite{Goode20121}.

Optimal forest management must consider the effects of the changing climate, because the time scale of forest development (decades) is the same as that on which climate change is predicted to operate (!CITE SOME REPORT that shows changes by late 21st century). Hence, optimal forest management will likely differ between future climate scenarios !CITE climate change forest management papers. Decisions that would once have resulted in optimal achievement of ecosystem services, now under different climatic conditions, may no longer do so. Without consideration of climate change, forest management plans may restrict forests' potential to provide ecosystem services most effectively. To determine which management practices will be optimal in the future, we must first understand how climate change will impact forests' ability to provide ecosystem services. For example, how many tons of carbon dioxide will the forest be capable of storing? How many acres of forest will still qualify as suitable habitat for a particular species? Many studies have considered these questions !CITE SOME PEOPLE.

However, previous studies have addressed the impact of climate change on forest ecosystem services in isolation. Because forests provide these ecosystem services in concert with one another (see, for example, \cite{toth2009finding}), we must also understand how climate impacts the tradeoffs that exist among them. Consider the simultaneous provision of wildlife habitat, carbon storage, and resistance to wildfire. How does an increase in any one service alter our ability to acquire an amount of another? Relationships such as a marginal sacrifice in one service for substantial improvement in another may no longer exist under a new climate. To better ensure the sustained provision of ecosystem services, we must understand how these tradeoffs evolve as a function of climate.

Here, I use a watershed in the Deschutes National Forest as a case study to determine how climate change impacts optimal forest management and the changes in tradeoffs among ecosystem services.