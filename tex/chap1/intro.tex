% ======== Introduction to Deschutes National Forest case study

\section{Introduction}
\label{sec:intro}

Many problems are multi-objective. Building aircraft requires attention to weight, price, AND SOME OTHER STUFF CITE. Hospitals seek to maximize patient care, number of patients seen, and minimize cost CITE. Forest managers aim to provide carbon sequestration, wildlife habitat, recreation, timber revenues, and protection from wildfire. Designing cars involves attention to objectives like horsepower, price, luxury, and fuel economy. Daily, we all try to maximize our own objective functions involving work, social interaction, fitness and others.
%Forests play an important role in global ecological, social, and economic processes. They provide ecosystem services such as carbon storage, purification of water and air, wildlife habitat, recreation opportunities, and generate raw materials for goods such as food and lumber \cite{daily1997ecosystem}. In managed forests, the extent to which forests provide these services depends in part on management practices. Forest management seeks to ensure the sustained provision of these ecosystem services \cite{cfrForestMgmt}.

Often, the objectives in these problems will conflict. When designing aircraft, as you decrease the weight, you drive price up. When hospitals maximize individual patient care, they are unable to see as many total patients. Forest managers increasing sequestered carbon drive up the severity of wildfires. Car manufacturers must generally sacrifice either price or luxury. We can only spend so many hours with friends and family and still keep our jobs.
%Like other ecosystems, forests will undergo changes as a result of the changing climate. Researchers anticipate new spatial distributions of tree species \cite{iverson1998predicting}, increased sediment delivery to streams \cite{Goode20121}, and increasing disturbance regimes such as wildfires, droughts, and insect infestations \cite{vose2012effects}. As this transformation occurs, the ability of forests to provide ecosystem services will change. Increased frequency of disturbance regimes will impact the ability of forests to store carbon \cite{bonan2008forests} and provide wildlife habitat \cite{mckenzie2004climatic}. Water supplies that rely on forests' filtration capabilities may be impacted by the rising sediment levels predicted by Goode \textit{et al.} \cite{Goode20121}.

To take full advantage of the resources at our disposal, optimal application of those resources is required. That is, understanding the balance between objectives let's you make better decisions. What if you adding just a bit of weight to the aircraft cut cost substantially? What if sacrificing just a bit of timber revenue entailed a huge increase in habitat for some species? Shaving off .1 seconds on the quarter-mile time on a new car only cost a hundredth of an mpg in the car you're building? For a decision maker to make the best decision, he or she must know the best compromises at their disposal. They must know if a solution exists that makes drastic improvements with minimal sacrifice. This requires an understanding of the conflicting relationships among the objectives.
%As a result, forest planners may need to consider the effects of the changing climate. Linder \cite{linder2000developing} has shown that optimal forest management strategies vary depending on which future climate scenario is assumed. To determine whether and how forest planners need to adapt to climate change, many studies have addressed the effects of climate change on isolated ecosystem services \cite{vose2012effects}\cite{bonan2008forests}\cite{mckenzie2004climatic}. However, forests provide these ecosystem services in concert with one another (see, for example, T{\'o}th and McDill \cite{toth2009finding}), and little research has been done to understand how climate change impacts the relationships among or the joint provision of ecosystem services. I posit that improvements in this understanding would enable better and more informed forest management decisions.

Numerous methods exist to attempt to quantify these conflicting relationships CITE. However, the inspiration for these methods is almost always the elimination of objectives from complex, many-objective problems in order to make them more computationally tractable. In our pursuit of better computational performance, we lost sight of why we have to deal with these problems in the first place: conflicting management objectives. We argue that no current conflict metric adequately addresses conflict to answer questions regarding the true essence of objective relationships.
%To improve our understanding of the impacts of climate change on the joint provision of ecosystem services, I propose the use of multi-objective mathematical optimization. Multi-objective optimization has been used in numerous forest management studies involving multiple ecosystem services \cite{schroder2016multi}\cite{toth2009finding}\cite{diaz2008making}\cite{borges2014addressing}. The application of multi-objective optimization in these studies reveals the relationships among ecosystem services. When the studies consider competing or ``bundled'' ecosystem services, multi-objective optimization produces a set of Pareto optimal management alternatives that reveal the trade-offs among the ecosystem services and allow for the quantification of conflict. In such cases, the competition among the ecosystem services prevents the simultaneous maximal provision of all ecosystem services. Instead, trade-offs must be made, where the provision of one or multiple ecosystem services must be reduced in order to increase provision of another. Timber revenues and wildlife habitat provide an example of bundled ecosystem services. In these cases, the optimal management strategy involves finding the best compromise alternatives.

Perhaps the most commonly used conflict metric, PEARSON, fails to detect the absence of conflict as many studies have defined it (mono inc). It also fails to capture any non-linear nuances between objectives. For instance if objs look like THIS GRAPH?, it fails to distinguish between the two. The SPEARMAN AND KENDALL and other commonly used rank-based metrics also fail to distinguish between those two scenarios. WHAT ABOUT MUTUAL INFO? Zitzler's delta error is a more nuanced indicator of conflict, but it does not provide a means of pair-wise objective comparison, is computationally complex, and is focused on relative relationships of objs and does not consider objective achievement.

We propose here a new method that addresses all of the issues described. It accurately captures the lack of conflict, is capable of handling non-linear correlation, is computationally simple, and considers objective achievement rather than just achievement relative to one another.

We demonstrate this new metric's use on a multi-objective case study in the Deschute National Forest. We also provide a consolidated discussion of conflict metrics for use in exact multi-objective optimization. This includes a discussion of Pareto frontier comparisons and how to interpret them for the sake of guidance to decision makers, not just for algorithm comparison.
%Heterogeneous forcing from climate change on forest ecosystem services will likely also change the trade-off relationships among ecosystem services. This has the potential to disrupt the balance among ecosystem services that define the compromise alternatives from which forest planers choose. In other words, under different climatic conditions, decisions that would once have resulted in a preferred balance among ecosystem services may no longer do so. Without consideration of climate change, forest management plans may restrict the potential for forests to provide ecosystem services most effectively. For instance, would the same amount of old-growth forest have to be converted to fire breaks to ensure a given reduction in fire hazard under various climate projections? Would carbon sequestration cost the same in terms of foregone timber revenues under a changing climate as it does today? To quantify the impacts of climate change on the trade-off relationships among and the joint provision of forest ecosystem services, I employ multi-objective optimization on a case study of the Drink Area, located in the Deschutes National Forest.

%ROADMAP
We define terms and layout notation first. Then we introduce the case study. Then we detail the results of the case study and the application of the metrics used here, both new and existing. And we conclude with closing remarks and a summary.
%In section \ref{sec:methods} I define conflict and relationships among ecosystem services and then describe a case study in which these are evaluated. Section \ref{sec:results} reports and discusses the results. Finally, I summarize and draw closing remarks in section \ref{sec:conclusion}.