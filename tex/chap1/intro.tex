% ======== Introduction to Deschutes National Forest case study

\section{Introduction}
\label{sec:intro}


%Forests play an important role in global ecological, social, and economic processes. They provide ecosystem services such as carbon storage, purification of water and air, wildlife habitat, recreation opportunities, and generate raw materials for goods such as food and lumber \cite{daily1997ecosystem}. In managed forests, the extent to which forests provide these services depends in part on management practices. Forest management seeks to ensure the sustained provision of these ecosystem services \cite{cfrForestMgmt}.

Like other ecosystems, forests will undergo changes as a result of the changing climate. Researchers anticipate new spatial distributions of tree species \cite{iverson1998predicting}, increased sediment delivery to streams \cite{Goode20121}, and increasing disturbance regimes such as wildfires, droughts, and insect infestations \cite{vose2012effects}. As this transformation occurs, the ability of forests to provide ecosystem services will change. Increased frequency of disturbance regimes will impact the ability of forests to store carbon \cite{bonan2008forests} and provide wildlife habitat \cite{mckenzie2004climatic}. Water supplies that rely on forests' filtration capabilities may be impacted by the rising sediment levels predicted by Goode \textit{et al.} \cite{Goode20121}.

As a result, forest planners may need to consider the effects of the changing climate. Linder \cite{linder2000developing} has shown that optimal forest management strategies vary depending on which future climate scenario is assumed. To determine whether and how forest planners need to adapt to climate change, many studies have addressed the effects of climate change on isolated ecosystem services \cite{vose2012effects}\cite{bonan2008forests}\cite{mckenzie2004climatic}. However, forests provide these ecosystem services in concert with one another (see, for example, T{\'o}th and McDill \cite{toth2009finding}), and little research has been done to understand how climate change impacts the relationships among or the joint provision of ecosystem services. I posit that improvements in this understanding would enable better and more informed forest management decisions.

To improve our understanding of the impacts of climate change on the joint provision of ecosystem services, I propose the use of multi-objective mathematical optimization. Multi-objective optimization has been used in numerous forest management studies involving multiple ecosystem services \cite{schroder2016multi}\cite{toth2009finding}\cite{diaz2008making}\cite{borges2014addressing}. The application of multi-objective optimization in these studies reveals the relationships among ecosystem services. When the studies consider competing or ``bundled'' ecosystem services, multi-objective optimization produces a set of Pareto optimal management alternatives that reveal the trade-offs among the ecosystem services and allow for the quantification of conflict. In such cases, the competition among the ecosystem services prevents the simultaneous maximal provision of all ecosystem services. Instead, trade-offs must be made, where the provision of one or multiple ecosystem services must be reduced in order to increase provision of another. Timber revenues and wildlife habitat provide an example of bundled ecosystem services. In these cases, the optimal management strategy involves finding the best compromise alternatives.

Heterogeneous forcing from climate change on forest ecosystem services will likely also change the trade-off relationships among ecosystem services. This has the potential to disrupt the balance among ecosystem services that define the compromise alternatives from which forest planers choose. In other words, under different climatic conditions, decisions that would once have resulted in a preferred balance among ecosystem services may no longer do so. Without consideration of climate change, forest management plans may restrict the potential for forests to provide ecosystem services most effectively. For instance, would the same amount of old-growth forest have to be converted to fire breaks to ensure a given reduction in fire hazard under various climate projections? Would carbon sequestration cost the same in terms of foregone timber revenues under a changing climate as it does today? To quantify the impacts of climate change on the trade-off relationships among and the joint provision of forest ecosystem services, I employ multi-objective optimization on a case study of the Drink Area, located in the Deschutes National Forest.

%ROADMAP
%In section \ref{sec:methods} I define conflict and relationships among ecosystem services and then describe a case study in which these are evaluated. Section \ref{sec:results} reports and discusses the results. Finally, I summarize and draw closing remarks in section \ref{sec:conclusion}.