% ======== Introduction to Deschutes National Forest case study

\section{Introduction}
\label{sec:intro}
 
Forests play an important role in global ecological, social, and economic processes. They provide ecosystem services such as carbon storage, purification of water and air, wildlife habitat, recreation opportunities, and generate raw materials for goods such as food and lumber \cite{daily1997ecosystem}. In managed forests, the extent to which forests provide these services depends in part on management practices. Optimal forest management seeks to ensure the sustained provision of these ecosystem services \cite{cfrForestMgmt}.

Like other ecosystems, forests will undergo changes as a result of the changing climate. Researchers anticipate new spatial distributions of tree species \cite{iverson1998predicting}, increased sediment delivery to streams \cite{Goode20121}, and increasing disturbance regimes such as wildfires, drought, and insect infestation \cite{vose2012effects}. As this transformation occurs, forests' ability to provide ecosystem services will change. Increased frequency of disturbance regimes will impact forests' ability to store carbon \cite{bonan2008forests} and provide wildlife habitat \cite{mckenzie2004climatic}. Water supplies that rely on forests' filtration capabilities may be impacted by the rising sediment levels predicted by Goode \textit{et al.} \cite{Goode20121}.

While many studies have addressed the impacts of climate change on forest ecosystem services in isolation\cite{vose2012effects}\cite{bonan2008forests}\cite{mckenzie2004climatic}, few have considered climate change's joint impact on multiple ecosystem services. Since forests provide these ecosystem services in concert with one another (see, for example, T{\'o}th and McDill \cite{toth2009finding}), it is necessary to understand how climate may impact the relationships among them. This is of particular interest to forest planners seeking the simultaneous provision of multiple ecosystem services, especially when those ecosystem services are bundled. ``Bundled'' ecosystem services are those that compete with one another and whose joint provision requires tradeoffs - the sacrifice of one ecosystem service in order to achieve more of another. Timber revenues and wildlife habitat provide an example of bundled ecosystem services.% (In contrast, ``stacked'' ecosystem services are those that are not in conflict, such as the provision of old growth forest and northern spotted owl habitat - an improvement in one ecosystem service also improves the others in the stack. Stacked ecosystem services do not present a management challenge, since the same management practices bring about positive change for all ecosystem services. \textit{IS THIS PARENTHESIZED TEXT NEEDED?})
%Does discussing bundles vs stacks just beg the question of whether bundles remain bundles or stacks remain stacks? Our research does not address this, so this is a dangerous question to beg.

In the case of bundled ecosystem services, the optimal management strategy is a balance in the provision of the ecosystem services. The ``goodness'' of the balance depends on the ecosystem services and the conflict that exists amongst them. In the case of weakly competing ecosystem services, a near-ideal management plan may exist such that all ecosystem services are attained near their maximal value. In the case of strongly competing ecosystem services, significant tradeoffs must be made to achieve the balance. As climate change is predicted to impact forests' ability to provide ecosystem services, I predict that the tradeoff relationships among the ecosystem services will also change as a result of heterogeneous forcing from climate change on forest ecosystem services.

As a result, forest managers may need to consider the effects of the changing climate, since %the time scale of forest development is of the same order as that on which climate change is predicted to operate \cite{ipcc2013climate}
optimal forest management will likely differ under alternative future climates \cite{linder2000developing}. Decisions that would once have resulted in a preferred balance among ecosystem services, now under different climatic conditions, may no longer do so. Without consideration of climate change, forest management plans may restrict forests' potential to provide ecosystem services most effectively. Consider, for example, a forest planner who must sacrifice old growth forest in order to remain below a threshold level of wildfire hazard. Does the same sacrifice of old growth forest continue to yield the required decrease in fire hazard under different climate conditions? If not, what new method will accomplish the fire hazard reduction with minimal sacrifice of old growth forest?

In this work, I use multi-objective mathematical optimization to quantify the changes in tradeoff relationships among ecosystem services using an area in the Deschutes National Forest as a case study. I posit generally that a better understanding of how climate change will impact tradeoff relationships will allow forest planners to make more informed management decisions going forward.

In section \ref{sec:methods} I describe the case study and the bundle of ecosystem services considered, discuss the selection of climate scenarios, introduce the mathematical optimization model, and describe how the solutions will be interpreted. Section \ref{sec:results} reports and discusses the results. Finally, I summarize and draw closing remarks in section \ref{sec:conclusion}.
%
%In the coming sections, I describe and motivate the case study, then motivate and describe the model.
%
%Then in the methods section, I had this note on the transitioning from case study to model:With these ESs, what are we going.
%
%There are these ESs. Then there's climate change. We're concerned  
%
%A good review to do - take an optimal solution to the NONE model (the values of the x vars, p vars, etc), and see what it would achieve in the E45 and E85 models (hoping to show that it would not have been a frontier point)
%
%
%\subsection{A hypothetical scenario}
%To understand how tradeoff relationships impact management strategies and ecosystem service provision, consider the following hypothetical scenario in which a forest planner manages for old forest habitat and wildfire hazard. Historically, serves as prime habitat for a threatened bird species. The forest manager's primary objective is the conservation of this species, but the manager also performs the minimum amount of harvests necessary to offset the costs of property ownership from timber sales.  To determine which stands to harvest, the manager ran an analysis, ignoring climate change, that suggested the harvest of a small set of stands near the northern perimeter of the forest as it was most accessible, served only as mediocre habitat for the bird species, and was dense with merchantable timber.
%
%Now, climate change is reducing the area of the forest that is suitable habitat for the threatened bird species. Much of the species' remaining habitat is the area on the northern perimeter of the forest that has traditionally been harvested. Simultaneously, longer growing seasons and warmer temperatures are increasing the timber stock of other areas of the forest, making the northern stands less relatively advantageous to harvest. The manager, unaware, continues harvests in the northern stands, damaging much of what remains of the bird species' habitat.
%
%With each ecosystem service modeled in isolation, this conflict is not fully understood, and optimal alternatives will not be discovered. Modeled in concert, however, the forest manager would have realized that these objectives do not strongly conflict with one another - if the manager were to move the harvests to the stands near the forest's eastern perimeter, his harvests would increase only slightly while also retaining nearly all of the bird species's suitable habitat. In other words, a marginal sacrifice in one ecosystem service allows for a significant improvement in another. Without knowledge of this tradeoff structure, the manager is unnecessarily impeding the bird species's climate-driven rangeshift.
%
%It is the impact of climate change on such tradeoff structures that I investigate in this work. I consider as a case study an area in the Deschutes National Forest known as the Drink Planning Area, or the Drink area, in which we attempt to maximize the provision of the following ecosystem services: area of habitat available to the northern spotted owl, water quality in the watershed, and reduction in fire hazard.