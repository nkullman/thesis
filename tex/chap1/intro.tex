% ======== Introduction to Deschutes National Forest case study

\section{Introduction}
 
Forests play an important role in global ecological, social, and economic processes. They provide ecosystem services such as carbon storage, purification of water and air, wildlife habitat, recreation opportunities, and generate raw materials for goods such as food and lumber \cite{daily1997ecosystem}. In managed forests, the extent to which forests provide these services depends in part on management practices. Optimal forest management seeks to ensure the sustained provision of these ecosystem services \cite{cfrForestMgmt}.

Like other ecosystems, forests will undergo changes as a result of the changing climate. Researchers anticipate new spatial distributions of tree species \cite{iverson1998predicting}, increased sediment delivery to streams \cite{Goode20121}, and increasing disturbance regimes such as wildfires, drought, and insect infestation \cite{vose2012effects}. As this transformation occurs, forests' ability to provide ecosystem services will change. New growing conditions will alter timber production \cite{forschungsanstalten2009adaptation}. REDO Novel temperatures and species compositions will alter forests' habitat suitability for inhabitants \cite{harding1997ecosystem}. Increased frequency of disturbance regimes will impact forests' ability to store carbon \cite{bonan2008forests} and provide wildlife habitat \cite{mckenzie2004climatic}. Water supplies that rely on forests' filtration capabilities may be impacted by the rising sediment levels predicted by \cite{Goode20121}.

Optimal forest management must consider the effects of the changing climate, because the time scale of forest development is of the same order as that on which climate change is predicted to operate \cite{ipcc2013climate}. Under alternative future climates, optimal forest management will likely differ \cite{linder2000developing}. Decisions that would once have resulted in optimal achievement of ecosystem services, now under different climatic conditions, may no longer do so. Without consideration of climate change, forest management plans may restrict forests' potential to provide ecosystem services most effectively.

Studies abound in the literature that have addressed the impact of climate change on forest ecosystem services in isolation. However, because forests provide these ecosystem services in concert with one another (see, for example, \cite{toth2009finding}), it is necessary to also understand how climate impacts the tradeoffs that exist among them. How does an increase in any one service alter our ability to acquire an amount of another? Relationships such as a marginal sacrifice in one service for substantial improvement in another may no longer exist under a new climate. To better ensure the sustained provision of ecosystem services, we must understand how these tradeoffs evolve as a function of climate.

$[$1Consider the following unfortunate, hypothetical scenario. Until recently, a particular forest served as prime habitat for bird species $A$. The forest's manager has also always managed the forest to generate enough revenue from timber sales to break even on property ownership.  A previous analysis that ignored climate change suggested that he harvest from a small area near the northern perimeter of the forest as it was most accessible and dense with merchantable timber. Now, however, climate change is quickly reducing the area of the forest that is suitable habitat for bird species $A$ while also increasing the timber stock of other areas of the forest. Due to the area's elevation, much of bird species $A$'s remaining habitat are those areas on the northern perimeter of the forest that have traditionally been harvested. By not simultaneously considering the ecosystem services, the forest manager is now unknowingly sacrificing large percentages of the potential habitat for bird species $A$ in order to maintain his timber revenues. Little does he know (bc he has not the advantage of using our research and knowing the tradeoff structure) that the tradeoff structure between these two objectives is weak. That is, they are not in strong conflict with one another. An analysis simultaneously considering both objectives would indicate that if he would shift his harvest to the eastern stands, he would sacrifice only a small amount in harvest revenues while securing the maximum possible habitat for bird species $A$, enabling an improved rangeshift to the more northern areas.$]$

-PROB DON'T NEED.What's more, the relationships between managed ecosystem services - that is, the situations in which one ecosystem service may be sacrificed in order to achieve more of another - are unknown under new climates. Under non-optimal conditions, this has the potential consequence of sacrificing some amount of an ecosystem service with no benefit in return. Under optimal conditions, sacrifices in one ecosystem service can be used to achieve more of another.

-In addition to the magnitude of change in any one ecosystem service, it is also unknown how climate change will impact the relationships between ecosystem services. That is, the landscape-level simultaneous achievement of multiple ecosystem services.

-an unnecessary sacrifice has been made in forgone revenue as a result

-To determine which management practices will be optimal in the future, we must first understand how climate change will impact forests' ability to provide ecosystem services. For example, how will climate change alter the ? How many acres of forest will still qualify as suitable habitat for a particular species? Many studies have considered these questions !CITE SOME PEOPLE.

Here, I use a watershed in the Deschutes National Forest as a case study to determine how climate change impacts optimal forest management and the changes in tradeoffs among ecosystem services.