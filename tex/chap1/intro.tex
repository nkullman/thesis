\section{Introduction}
 
Forests are crucial to global ecological, social, and economic processes. They provide ecosystem services such as carbon storage, purification of water and air, wildlife habitat, recreation opportunities, and generate raw materials for goods such as food and lumber \cite{daily1997ecosystem}. Many forests are actively managed, either by public or private entities \cite{white2002owns}. These managing entities prioritize the ecosystem services to be provided by the land. The extent to which forests can provide these services depends on management practices. The ultimate goal in forest planning is to ensure the sustained provision of these ecosystem services.

Similar to all other ecosystems, forests are predicted to undergo complex changes as a result of the changing climate. We anticipate new spatial distributions of tree species \cite{iverson1998predicting}, increased sediment delivery to streams \cite{Goode20121}, and increasing disturbance regimes such as wildfires, drought, and insect infestation \cite{vose2012effects}. As this transformation occurs, forests’ ability to provide ecosystem services will change. For instance, increased frequency of wildfires will impact forests’ ability to store carbon \cite{bonan2008forests} and provide habitat for wildlife \cite{mckenzie2004climatic}. Water supplies that rely on forests’ filtration capabilities may be impacted by the rising sediment levels predicted by \cite{Goode20121}.

As forest management often plans decades into the future, proper planning must include the effects of changing climate. Under new climatic conditions, decisions that once would have resulted in levels of optimal achievement of ecosystem services may no longer do so. Without proper correction we may be managing forests in a way that restricts their potential to provide ecosystem services most effectively. To determine which management practices will be optimal in the future, we must first understand how climate change will impact forests’ ability to provide ecosystem services. For example, how many tons of carbon dioxide will the forest be capable of storing? How many acres of forest will still qualify as suitable habitat for a particular species? Many studies have considered these questions but address ecosystem services in isolation.

Because forests provide these ecosystem services in concert with one another (see, for example, \cite{toth2009finding}), we must also understand how climate impacts the tradeoffs that exist among them. Consider the simultaneous provision of wildlife habitat, carbon storage, and resistance to wildfire. How does an increase in any one service alter our ability to acquire an amount of another? Relationships such as a marginal sacrifice in one objective for substantial improvement in another may no longer be as advantageous under a new climate. To properly adapt management decisions and ensure the sustained provision of ecosystem services, we must understand how these tradeoffs evolve as a function of climate.
