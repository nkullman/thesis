% ======== Introduction to Deschutes National Forest case study

\section{Introduction}
\label{sec:intro}

% Many problems are multi-objective
Many tasks in resource allocation are multi-objective. The design of aircraft involves balancing cost and efficiency \cite{wang2014multi}. Hospitals seek to manage personnel and equipment in order to maximize patient throughput while minimizing cost and required back-up \cite{hutzschenreuter2009evolutionary}. Food production balances processing time with nutrient retention \cite{sendin2010efficient}. Forest managers aim to provide carbon sequestration and wildlife habitat while also maximizing timber revenues \cite{toth2013ecosel}.

% Objs in these problems may conflict
Given a set of solutions to one of these resource allocation problems, a decision maker chooses one to enact. Often, no one solution simultaneously optimizes all objectives, and the decision maker must therefore choose a solution that represents a preferred compromise among them. In such cases, there is some amount of conflict among the objectives. This is in contrast to compatible or harmonious objective relationships in which the objectives improve simultaneously.

% This is what conflict looks like
In the case of aircraft design, cost and efficiency conflict with one another, since more efficient design details tend to cost more. Similarly, hospitals may increase patient throughput by increasing the number of doctors available, but this decision would increase costs. Food production engineers can maximize nutrient retention by reducing the temperature at which processing occurs, but this would lengthen the time required to reach acceptable microbiological levels. Forest managers can maximize timber revenue by removing large old-growth timber, but this would reduce the available wildlife habitat.

% We need to understand conflict so we can make good decisions
While the preferred solution may vary by decision maker, a rational decision maker will prefer one which is Pareto efficient; that is, a solution in which no objective can be improved without compromising another. Multi-objective optimization affords the knowledge of such solutions and can help guide the decision maker by revealing where objectives can be achieved simultaneously and where they conflict. Having access to the set of Pareto efficient solutions may also help the decision maker locate solutions where compromises in one objective allow outweighing improvements in another. For instance the forest manager may discover that forgoing small amounts of timber revenues allows for the sequestration of significantly more carbon. Or the hospital may be able to increase patient throughput substantially if they hire one additional oncologist.
Regardless of whether a decision maker selects a solution providing such gains, the awareness of these relationships enables more informed decision making.

% How might conflict compare across multi-objective systems?
In addition to studying conflict within a system, we may further consider the situation in which a decision maker oversees multiple systems, each with its own set of Pareto efficient solutions. This could be the case for a manager overseeing multiple hospitals or multiple food processing facilities. Alternatively, each system could correspond to a different scenario, such as a forest manager analyzing resource allocation under various realizations of climate change. In such instances, understanding the changes in the conflict relationships between systems may benefit the decision maker, allowing them to ask questions such as: How does the relationship between carbon sequestration and timber revenues vary under different climate change scenarios? Do all hospitals require the same increase in cost to increase patient throughput?

% We don't know, bc no one has asked this question. We're going to, and here's how. Requires a new metric
To date, the multi-objective optimization literature has not addressed conflict in these system-level questions. We do so for the first time here, laying a foundation for quantitative conflict analysis. To perform this investigation, we draw on measures commonly used in the field of evolutionary multi-objective optimization (EMO), including various Pareto set indicators and correlation measures. Researchers in EMO use the Pareto set indicators to measure the performance of algorithms that approximate the optimal Pareto set \cite{zitzler2003performance}. The correlation measures are used as an aid to increase the computational tractability of the multi-objective problems encountered in EMO \cite{brockhoff2006all}. Here we adapt these measures to better understand conflicting management objectives across systems. We also develop a new metric for quantifying the conflict between a pair of objectives. The new pairwise conflict metric developed here improves on other commonly used pairwise conflict metrics such as the Pearson and Spearman coefficients. Unlike any current metric, the one we propose can capture mutual objective achievement and accurately identify the lack of conflict between objectives. We demonstrate the novel utility of the existing and proposed conflict metrics on a multi-objective scenario-based case study in the Deschutes National Forest.

% Ask the new conflict questions RE system-level conflict and demonstrate how 
In the upcoming sections, we first define terminology. Then we detail the case study and present its results including the application of the new and existing metrics. We conclude with a summary and suggestions for future research.