% ======== Introduction to Deschutes National Forest case study

\section{Introduction}
\label{sec:intro}

% Many problems are multi-objective
Many tasks in resource allocation are multi-objective. The design of aircraft involves balancing cost and efficiency \cite{wang2014multi}. Hospitals seek to manage personnel and equipment in order to maximize patient throughput while minimizing cost and required back-up sources \cite{hutzschenreuter2009evolutionary}. Food production balances processing time with nutrient retention \cite{sendin2010efficient}. Forest managers aim to provide carbon sequestration and wildlife habitat while also benefiting from timber revenues \cite{toth2013ecosel}.

% Objs in these problems have varying relationships
Given a set of solutions to one of these resource allocation problems, a decision maker chooses one to enact. Often, no one solution simultaneously optimizes all the objectives, and the decision maker must therefore choose a solution that represents a preferred balance among them. In such cases, there is some amount of conflict among the objectives. This is in contrast to compatible or harmonious objective relationships in which the objectives improve simultaneously.

% Examples of those relationships
In the case of aircraft design, cost and efficiency conflict with one another, since more efficient efficient design details tend to cost more. Similarly, hospitals may increase patient throughput by increasing the number of doctors available, but this decision would increase costs. Food production engineers can maximize nutrient retention by reducing the temperature at which processing occurs, but this would lengthen the time required to reach acceptable micriobiological levels. While the forest manager maximizes timber revenue by removing large old-growth timber, this would reduce the available wildlife habitat.% In contrast, the forest manager may be able to simultaneously optimize objectives such as wildlife habitat and maintenance costs by performing no actions. The hospital may be able to improve patient throughput and SOMETHING ELSE. These are examples of compatible objectives.

% Understanding conflict, compatibility, and obj relationships is important - 1
While the preferred solution may vary by decision maker, a rational decision maker will prefer one which is Pareto efficient; that is, a solution in which no objective can be improved without compromising another. Knowledge of such solutions is a benefit gained from multi-objective optimization. Knowing the set of Pareto efficient solutions can help guide the decision maker by revealing where objectives can be achieved simultaneously and where they conflict. The decision maker may also use it to locate solutions where compromises in one objective allow outweighing improvements in another. For instance the forest manager may discover that forgoing small amounts of timber revenues allows for the sequestration of significantly more carbon. Or the hospital may be able to increase patient throughput substantially if they hire one additional oncologist.
Regardless of whether the decision maker selects a solution that provides the gains described, the improved understanding of these conflict relationships and trade-offs enables more informed decision making.

% Understanding conflict, compatibility, and obj relationships is important - 2
We may further consider the situation in which a decision maker considers multiple sets of solutions. This could be the case for a manager overseeing multiple hospitals or multiple food processing facilities. Alternatively, each set could correspond to a different scenario, such as a forest manager analyzing resource allocation under various potential realizations of climate change. In such instances, understanding the changes in the conflict relationships between the sets of solutions may benefit the decision maker. How does the relationship between carbon sequestration and timber revenues differ under the climate scenarios? Do all hospitals have the same degree of conflict between patient throughput and cost?

To date, addressing such questions has not been covered by the multi-objective optimization literature. We do so for the first time here. To perform this investigation, we draw on methods commonly used in the field of evolutionary multi-objective optimization (EMO). Whereas in EMO they are used to increase problems' computational tractability CITE or assess the quality of heuristics used to solve them CITE, here we apply them to better understand conflicting management objectives, especially in the case of comparing multiple Pareto efficient frontiers. We also make modifications to the methods as necessary and develop a new metric for studying the conflict between a pair of objectives.

% We currently know how to measure some of these things.
% But there are gaps in our ability to quantify certain things (pairwise conflict.)
Numerous methods exist to attempt to quantify these conflicting relationships CITE. Some measures capture holistic conflict HYPERVOL/EPSILON ERROR. These are important and useful. Often we want more detail on pairwise conflict too. Other measures exist to provide the decision maker with the ability to drill down into this. However, the motivation for these pairwise methods is almost always the elimination of objectives from complex, many-objective problems in order to make them more computationally tractable. For these methods, In our pursuit of better computational performance, we lost sight of why we have to deal with these problems in the first place: conflicting management objectives. We argue that no current pairwise conflict metric adequately measures conflict to answer questions regarding objective relationships.

% Here is where the gaps in that are
Perhaps the most commonly used pairwise conflict metric, PEARSON, fails to detect the absence of conflict as many studies have defined it (mono inc). For instance if objs look like THIS GRAPH?, it fails to distinguish between the two. The SPEARMAN AND KENDALL and other commonly used rank-based metrics also fail to distinguish between those two scenarios. WHAT ABOUT MUTUAL INFO? Zitzler's delta error is a more nuanced indicator of conflict, but it does not provide a means of pair-wise objective comparison, is computationally complex, and is focused on relative relationships of objs and does not consider objective achievement.

% Here is a way we can fill the gaps
We propose here a new pairwise conflict metric that addresses all of the issues described. It accurately detects the lack of conflict, is capable of handling non-linear correlation, is computationally simple, and considers objective achievement rather than just achievement relative to one another.

% I want to show how to use the new thing, and the things we already know how to compute in a new way.
We demonstrate this new metric's use on a multi-objective case study in the Deschutes National Forest. We also provide a consolidated discussion of conflict metrics for use in exact multi-objective optimization. This includes a discussion of Pareto frontier comparisons and how to interpret them for the sake of guidance to decision makers, not just for algorithm comparison.

% Here's where the rest of this paper is going.
We define terms and layout notation first. Then we introduce the case study. Then we detail the results of the case study and the application of the metrics used here, both new and existing. And we conclude with closing remarks and a summary.