% ======== Introduction to Deschutes National Forest case study

\section{Introduction}
 
Forests play an important role in global ecological, social, and economic processes. They provide ecosystem services such as carbon storage, purification of water and air, wildlife habitat, recreation opportunities, and generate raw materials for goods such as food and lumber \cite{daily1997ecosystem}. In managed forests, the extent to which forests provide these services depends in part on management practices. Optimal forest management seeks to ensure the sustained provision of these ecosystem services \cite{cfrForestMgmt}.

Like other ecosystems, forests will undergo changes as a result of the changing climate. Researchers anticipate new spatial distributions of tree species \cite{iverson1998predicting}, increased sediment delivery to streams \cite{Goode20121}, and increasing disturbance regimes such as wildfires, drought, and insect infestation \cite{vose2012effects}. As this transformation occurs, forests' ability to provide ecosystem services will change. Increased frequency of disturbance regimes will impact forests' ability to store carbon \cite{bonan2008forests} and provide wildlife habitat \cite{mckenzie2004climatic}. Water supplies that rely on forests' filtration capabilities may be impacted by the rising sediment levels predicted by \cite{Goode20121}.

Optimal forest management must consider the effects of the changing climate, because the time scale of forest development is of the same order as that on which climate change is predicted to operate \cite{ipcc2013climate}. Optimal forest management will likely differ under alternative future climates \cite{linder2000developing}. Decisions that would once have resulted in optimal achievement of ecosystem services, now under different climatic conditions, may no longer do so. Without consideration of climate change, forest management plans may restrict forests' potential to provide ecosystem services most effectively.

Many studies have addressed the impacts of climate change on forest ecosystem services in isolation\cite{vose2012effects}\cite{bonan2008forests}\cite{mckenzie2004climatic}. However, because forests provide these ecosystem services in concert with one another (see, for example, \cite{toth2009finding}), it is necessary to also understand how climate impacts the tradeoffs that exist among them. How does an increase in any one ecosystem service alter our ability to acquire an amount of another? Relationships such as a marginal sacrifice in one service for substantial improvement in another may no longer exist under a new climate. To better ensure the sustained provision of ecosystem services, we must understand how these tradeoffs evolve with climate.

\subsection{A hypothetical scenario}
To understand how tradeoff relationships impact management strategies and ecosystem service provision, consider the following hypothetical scenario in which a forest serves as prime habitat for a threatened bird species. The forest manager's primary objective is the conservation of this species, but the manager also performs the minimum amount of harvests necessary to negate the costs of property ownership from timber sales.  To determine which stands to harvest, the manager ran an analysis, ignoring climate change, that suggested the harvest of a small set of stands near the northern perimeter of the forest as it was most accessible, served only as mediocre habitat for the bird species, and was dense with merchantable timber.

Now, climate change is reducing the area of the forest that is suitable habitat for the threatened bird species. Much of the species' remaining habitat is the area on the northern perimeter of the forest that has traditionally been harvested. Simultaneously, longer growing seasons and warmer temperatures are increasing the timber stock of other areas of the forest, making the northern stands less relatively advantageous to harvest. The manager, unaware, continues harvests in the northern stands, damaging much of what remains of the bird species' habitat.

With each ecosystem service modeled in isolation, this conflict is not fully understood, and optimal alternatives will not be discovered. Modeled in concert, however, the forest manager would have realized that these objectives do not strongly conflict with one another - if the manager were to move the harvests to the stands near the forest's eastern perimeter, his harvests would increase only slightly while also retaining nearly all of the bird species's suitable habitat. In other words, a marginal sacrifice in one ecosystem service allows for a significant improvement in another. Without knowledge of this tradeoff structure, the manager is unnecessarily impeding the bird species's climate-driven rangeshift.

It is the impact of climate change on such tradeoff structures that I investigate in this work. I consider as a case study an area in the Deschutes National Forest known as the Drink Planning Area, or the Drink area, in which we attempt to maximize the provision of the following ecosystem services: area of habitat available to the northern spotted owl, water quality in the watershed, and reduction in fire hazard.