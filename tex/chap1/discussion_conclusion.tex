\section{Discussion \& Conclusion}
We demonstrated a ne

\subsection{New conflict metric} - should this still stay here? probably. Try to keep wording non-specific to the case study, since some readers will have skipped it. Go with something like "We saw in the case study that the conflict metric was able to sniff out nuanced differences between objectives across multi-obj systems..." or soemthing else awesome like that.

\paragraph{Pros}
We saw that fire-sed relationship is very similar across climate scenarios, and the metric captures this. However, it even succeeds in capturing the nuanced "gets worse with climate change" even though its slight. Man o man.
Also, it gives a higher conflict value for fire-sed than it does to either of the other obj pairs, which is good, bc there's more conflict there. The others are more scattered, and it gives low values for these.
However, it might be that it isn't as good as sniffing out conflict among minimally conflicting objective pairs. It tends to vary a lot for the scattered ones, although that can be justified too, since the sols in that one plot show that E85 spends less time near the bad values, so really, I mean, it's OK.

\paragraph{Cons}
When looking at all pairwise conflict metrics for a frontier, their combination does not let you make deductions of the whole system level conflict.
Weights average objective achievement perhaps too much. Frontiers with a bunch of solutions nearer the ideal are given less conflict. perhaps that's OK though.
%SHORTCOMING: THE FACT THAT THE COMBNIATION OF C\_IJ MEASURES DOES NOT PROVIDE ANY %EVIDENCE FOR WHY A HYPERVOL MIGHT BE BIG OR SMALL


TO TALK ABOUT SOMEWHERE:
Why the disparity in number of solutions?
Why E85 achieve relatively more NSO habitat and relatively less FH on average than the others (fig \ref{fig:pairplotNSOSed} and sim for NSO-FH)?
The increased range in achievable values for the objs in the case of climate change leads to results that show tighter clustering closer to ideal, which weights the conflict towards "less". Basically, more extreme extremes make most other values seem more normal(?).

INVESTIGATE MORE:
What's the difference in actual acreage of NSO hab vs diffs due to clustering discounts? Made that map, but don't want to talk about unless I'm encouraged (Sandor...) to talk about it. Make sure this sentence doesn't make it into my next draft.
NUM SOLUTIONS = bc of the DQing of NSO hab as a result of actions, there are more combos of ways to do that and pvar values as a result. And so the model has to make more decisions about whether to make something owl habitat or instead do a treatment. And this leads to many more results that we didn't have before. And lots of constraints involve the pvars and qvars, and so this would mean harder to solve (which we saw but didn't talk about here). Also the 