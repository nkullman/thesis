\section{Discussion \& Conclusion}
% We did good
We used a case study of the impact of climate change on the joint provision of forest ecosystem services to successfully demonstrate the utility of a new measure of pairwise objective conflict and to demonstrate a new application of existing conflict measures in the quantification of conflict within and among multi-objective systems.

% We did good in something that was hard to do good in
We argue that the case study served as a rigorous first test of the process and the conflict measures, because there was little overall conflict in these systems and the differences in relative objective achievement across climate scenarios were not great. For instance, in the case study, the hypervolume for each climate scenario was relatively large, with solutions occupying over 80\% of the objective space in all cases. In addition, in all but one pairwise objective comparison, it was difficult to discern any distinct conflict relationship between the objectives. As a result, our proposed conflict metric and the hypervolume indicators were required to detect subtle differences in conflict, which they did successfully. Should the differences in objective achievement between climate scenarios have been more pronounced, or should the objectives have been in greater conflict with one another, we suspect the utility of these measures and the process we demonstrated here would only increase.

% But who's to say that it will always do good. It may not. You'd want to also consider other scenarios, like this food processing one
Of course, our success here is not a guarantee of future success. In consideration of different climate change scenarios, different ecosystem services, or a different study area, the conflict measures may prove less useful. The application of this quantitative conflict analysis should also be tested in other multi-objective systems.

Consider again the manager of the food processing facility, this time with looming regulatory changes on the horizon, such as a change in the maximum allowable microbiological levels. For each of a number of such levels, the manager may consider the balance in processing time and nutrient retention. Would the hypervolumes always report greater joint objective provision under larger allowable levels? Or in the case of the manager overseeing multiple hospitals, how does the pairwise conflict between operating cost and patient throughput vary between them?

% We think our thing would continue to do good, bc, again, it did good here. It was so great.
Based off our results, we believe that our new conflict measure and the process we suggest would be useful to the managers in these cases as well. As we saw, the proposed conflict measure was successful in being able to identify which objective pairs were most in conflict. We also saw that the hypervolumes were successful in detecting increasing system-level conflict under different environmental conditions. These variations in conflict were supported by the underlying model data. 

% But it's not perfect, you know?
However, the tool is not without its shortcomings. We first note that differences in low-level-of-conflict scenarios should be investigated more thoroughly, bc Cij may vary widely when both its components are small. In these cases, the individual components have more influence, and so slight variations in them can lead to large diffs in Cij. Also, when looking at the combination of all Cij measures for objs within a frontier, their combination does not give you details about how the hypervol measure may vary. For instance, the lowest hypervol scenario for us also had the lowest sum of Cijs.

% So we need to do more research, more case studies, and we need to try to make our tool better.
In sum, more case studies should be done using this conflict measure and the others old ones we used here. Additionally, refinements to the conflict measure should be investigated, especially in the case where ran correlation and avg distance to ideal are both mid-range.

% But suffice it to say, we're awesome. Over and out.	
But all we've done is great, really. The end.